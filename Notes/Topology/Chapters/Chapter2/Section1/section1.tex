%----------------------------------------------------------------------------------------
%	SECTION 1.1
%----------------------------------------------------------------------------------------

\section{The Product Topology.}

We now explore more about the product topology.

\begin{definition}
    LEt $J$ be an indexed set, and let  $X$ be a set. We define a  \textbf{$J$-tuple} of elements of
    $X$ to be a map $x:J \rightarrow X$, where if  $\alpha \in J$, then  $x(\alpha)=x_{\alpha}$, and
    we call it the \textbf{$\alpha$-th coordinate} of $x$. We write  $(x_{\alpha})_\alpha \in J$, or
just simply $(x_{\alpha})$
\end{definition}

\begin{definition}
Let $\{A_{\alpha}\}$ be an indexed family, and let $X=\bigcup_{\alpha \in J}{A_{\alpha}}$. We define
the \textbf{cartesian product} of $\{A_{\alpha}\}$, $\prod_{\alpha \in J}{A_\alpha}$ to be the set of
all $J$-tuples  $(x_{\alpha})$ of elements of $X$, where  $x_{\alpha} \in A_{\alpha}$
\end{definition}

\begin{theorem}\label{2.1.1}
    Let $\{X_{\alpha}\}$ be a family of topological spaces, and consider the cartesian product
    $\prod{X_{\alpha}}$. Then the collection of all cartesian products $\prod{U_{\alpha}}$, where
    $U_{\alpha}$ is open in $X_{\alpha}$, for all $\alpha$, forms a basis for the topology on
    $\prod{X_{\alpha}}$.
\end{theorem}
\begin{proof}
    Clearly $\prod{X_{\alpha}}$ itself is a basis element by the first condition. Now consider
    $\prod{U_{\alpha}}$ and $\prod{V_{\alpha}}$, then $\prod{U_{\alpha}} \cap
    \prod{V_{\alpha}}=\prod{U_{\alpha} \cap V_{\apha}}$, which is also a basis element.
\end{proof}

\begin{definition}
Let $\{X_{\alpha}$ be a family of topological spaces, and take as basis the collection of all
    products $\prod{U_{\alpha}}$ where $U_{\alpha}$, where $U_{\alpha}$ is open in $X_{\alpha}$.
    We call the topology generated by this basis the \textbf{box topology} on $\prod{X_{\alpha}}$.		
\end{definition}

\begin{definition}
    Let $\pi_{\beta}:\prod{X_{\alpha}} \rightarrow X_{\beta}$ be defined by
    $\pi_{\beta}((x_\alpha))=x_\beta$. We call this map the \textbf{projection mapping} of
    $\prod{X_{\alpha}}$ onto $X_{\beta}$		
\end{definition}

\begin{theorem}\label{2.1.2}
    Let $\Sc_{\beta}=\{\pi^{-1}_{\beta}(U_{\beta}):U_{\beta} \text{ is open in } X_{\beta}\}$, and let
    $\Sc=\bigcup{\Sc_{\beta}}$. Then $\Sc$ forms the basis for a topology on  $\prod{X_[\alpha]}$.
\end{theorem}
\begin{proof}
    Since $U_{\beta}$ is open in $X_{\beta}$, $\pi^{-1}_{\beta}(U_{\beta})) \subseteq
    \prod{X_\alpha}$. Taking $\bigcup{\Sc}$, we get that $\bigcup{\pi^{-1}_{\beta}(U_{\beta}))} =
    \prod{X_\{beta}$ for all $\beta$. Thus  $\Sc$ is a subbasis.
\end{proof}


\begin{definition}
    Let $\pi_{\beta}$ be a projection mapping of $\prod{X_{\alpha}}$ onto $X_[\beta]$, and take as
    subbasis the collection of all $\pi^{-1}_{\beta}(U_{\beta})$ where $U_{\beta}$ is open in
    $X_{\beta}$. We call the topology generated by this subbasis the \textbf{product space
    topology}, or more generally the \textbf{product topology} on $\prod{X_{\alpha}}$.		
\end{definition}
