%----------------------------------------------------------------------------------------
%	SECTION 1.1
%----------------------------------------------------------------------------------------

\section{The Quotient Topology}

\begin{definition}
    Let $X$ and  $Y$ be topological spaces, and let  $p:X \rightarrow Y$ be onto. We say that  $p$
    is a \textbf {quotient map} if a subset $U \subseteq Y$ is open if and only if  $p^{-1}(U)
    \subseteq X$ is open. We say that a subset $C$ is \textbf {saturated} with respect to $p$
    if for every $p^{-1}(\{y\})$ that intersects $C$,  $p^{-1}(\{y\}) \subseteq C$; that is $C \cap
    p^{-1}(\{y\})=p^{-1}(\{y\})$.
\end{definition}

\begin{definition}
    Let $f:X \rightarrow Y$ be a map, with $X$ and  $Y$ topological spaces. We say that  $f$ is an
    \textbf {open map} if for each open subset $U \subseteq X$,  $f(U) \subseteq Y$ is also open. We say $f$ is a
    \textbf {closed map} if for each closed subset $U \subseteq X$,  $f(U) \subseteq Y$ is closed.
\end{definition}

\begin{lemma}\label{4.4.1}
    If $p:X \rightarrow Y$ is a continuous map of  $X$ onto  $Y$, for topological spaces  $X$ and
    $Y$, that is either open or closed, then  $p$ is a quotient map.
\end{lemma}
\begin{remark} 
    A quotient map need not be open nor closed.		
\end{remark}

\begin{example}
    \begin{enumerate}[label=(\arabic*)]
        \item Let $X$ be the subspace  $[0,1] \cup [2,3]$ and $Y$ be the subspace  $[0,2]$ in $\R$,
            and defined  $p:X \rightarrow Y$ by  $p(x)=\begin{cases}
                                                x, \text{ } x \in [0,1] \\
                                                x-1, \text{ } x \in [2,3] \\
                                            \end{cases}$.
        We have that $p$ is continuous onto, by the pasting lemma, furthermore,  $p$ is also closed;
        hence  $p$ is a quotient map.  $p$ is not open however, as  $p([0,1])=[0,1]$ is closed in
        $Y$.

        Now if  $A$ is the subspace  $[0,1) \cup [2,3]$ of $X$, then  $\:A \rightarrow Y$ is
        continuous onto, but fails to be closed. So  $p|_A$ is not a quotient map, depsite the fact
        that  $[2,3]$ is open in $A$ and saturated with respect to  $p|_A$.

        \item Let $\pi_1:\R \times \R \rightarrow \R$ be the projection map that takes $x \times y
            \rightarrow x$. Clearly  $\pi_1$ is continuous onto, and $\pi_1$ is also open as
            $\pi_1(U \times V)=U$ which is open in $Y$; hence  $\pi_1$ is a quotient map. Now
            consider the closed set $C=\{x \times y:xy=1\}$ in $\R \times \R$.
            $\pi_1(C)=\com{\R}{\{0\}}$ which is not closed.
    \end{enumerate}
\end{example} 

\begin{theorem}\label{2.4.2}
    Let $X$ be a topological space, and  $A$ a set, and let  $p:X \rightarrow Y$ be onto. Define
    $\Tc$ to be the collection of subsets  $U$ of  $A$ for which  $p^{-1}(U)$ is open in $X$. Then
    $\Tc$ is a unique topology for which  $p$ is a  quotient map.
\end{theorem}
\begin{proof}
    Clearly $\emptyset,A \in \Tc$, for  $p^{-1}(\emptyset)=\emptyset$ and $p^{-1}(A)=X$. Now let
    $\{U_{\alpha}\}$ and $\{U_i\}_{i=1}^{n}$ be collections of subsets of $A$. Then
        \begin{equation*}
            p^{-1}(\bigcup{U_{\alpha}})=\bigcup{p^{-1}(U_{\alpha})}
        \end{equation*}
    and 
        \begin{equation*}
            p^{-1}(\bigcap_{i=1}^{n}{U_i})=\bigcap_{i=1}^{n}{p^{-1}(U_i)}.
        \end{equation*}
    Thus $\Tc$ is a topology on  $A$. Now notice that this also makes  $p$ into a quotient map.

    Now suppose that there is another topology $\Tc'$ for which  $p$ is a quotient map.  Clearly $\Tc
    \subsesubseteq \Tc'$, now if $p$ is open, then for $U$ open in $X$,  $p(U)$ is open in $A$,
    hence so are their preimages, and since $p$ is continuous and onto this makes $\Tc' \subseteq
    \Tc$, thus $\Tc$ is unique. Likewise, by similar reasoning with closed sets, if  $p$ is closed,
     $\Tc$ is still unique.
\end{proof}

\begin{definition}
    If $X$ is a topologicas space, and  $A$ a set, and  $p:X \rightarrow A$ is onto, then there is
    exactly one topology  $\Tc$ on  $A$ for which  $p$ is a quotient map. We call this topology the
    \textbf {quotient topology} on $A$ induced by  $p$.
\end{definition}

\begin{example}
    Let $p:\R \rightarrow A$, with  $A=\{a,b,c\}$ be defined by
        \begin{equation*}
            p(x)=\begin{cases}
                    a, \text{ } x>0 \\
                    b, \text{ } x<0 \\
                    c, \text{ } x=0 \\
                \end{cases}
        \end{equation*}
    Then the quotient topology on $A$ is the topology  $\Tc=\{\emptyset, \{a\}, \{b\}, \{a,b\},A\}$.
    
    \begin{figure} 
        \centering
        \includegraphics[scale = 0.2]{Figures/Chapter2/quotientOnAbc.png}
        \caption{The quotient topology on $A=\{a,b,c\}$ induced by $p$.}
    \end{figure}
\end{example} 

\begin{definition}
    Let $X$ be a topological space and  $X/p$ be a partition of  $X$ into disjoint subsets whos
    disjoint union is  $X$. Let  $p:X \rightarrow X/p$ be onto such that  $p:x \rightarrow U$ if  $x
    \in U$. We call  $X/p$ in the quotient topology induced by $p$ the \textbf{quotient space} of $X$, or the 
    \textbf{decomposition space} of $x$.
\end{definition}

We can take an equivalence relation $\sim$ on  $X$ by taking  $x \sim y$ if  $x,y \in U$ for  $U \in
X/p$. Then the quotient space is the collection of equivalence classes of $X$; that is we can think
of obtaining $X^*$ by ``identify'' those pairs of equivalent points. Similarly, we can also describe
the quotient space $X/p$ by noting a subset $U$ of equivalence classes where $p^{-1}(U)=\bigcup_{V
\in U}{V}$ is the union of all equivalence classes in $U$. We will denote the quotient space by
$X/p$ or  $X/\sim$.

\begin{figure} 
    \centering
    \includegraphics[scale = 0.2]{Figures/Chapter2/diskSphere.eps}
\end{figure}
