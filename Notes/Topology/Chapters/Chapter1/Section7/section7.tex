%----------------------------------------------------------------------------------------
%	SECTION 1.5
%----------------------------------------------------------------------------------------

\section{Continuous Functions.}

\begin{definition}
    Let $X$ and  $Y$ be topological spaces. We say that a mapping $f:X
    \rightarrow Y$ is  \textbf{continuous} if for each open set $V$ in  $Y$,
    $f^{-1}(V)$ is open in $X$.
\end{definition}

Now if $f:X \rightarrow Y$ is continuous, the for every open set  $V$ of  $Y$,
$f^{-1}(V)$ is open in  $X$. Now suppose that  $\Bc$ is a basis of  $Y$, then
$V=\bingcup{B_{\alpha}}$, hence
$f^{-1}(\bingcup{B_{\alpha}})=\bingcup{f^{-1}{B_{\alpha}}}$, which is open in
$X$, thus  $B_{\alpha}$ must also be open in  $X$. 

Similarly, if $\Sc$ is a subbasis of  $Y$, then for any basis element  $B$ of
$Y$,  $B=\bigcap_{i=1}^{n}{S_i}$, which then implies that
$f^{-1}(B)=\bigcap_{i=1}^{n}{f^{-1}(S_i)}$, thus  $S_i$ is also open in  $X$ for
 $1 \leq i \leq n$.

\begin{example}
    \begin{enumerate}[label=(\arabic*)]
        \item Let $f:\R \rightarrow \R$ be a continuous realvalued function.
            THen for each open interval  $I \subseteq \R$,  $f^{-1}(I)$ is an
            open interval in  $\R$, so take  $x_0 \in \R$ and $\epsilon>0$, and
            let $I=(f(x_0)-\epsilon,f(x_0)+\epsilon)$, then since $ x_0 \in
            f^{-1}(I)$, there is a basis $(a,b) \subseteq f^{-1}(I)$ about
            $x_0$. Then take  $\delta=\min\{x_0-a,x_0-b\}$, then $x \in (a,b)$
            whenever $0<|x-x_0|<\delta$, and we get that $f(x) \in I$, that is,
            $|f(x)-f(x_0)|<\epsilon$. This is the definition of continuity
            defined in the real analysis. We can prove that the converse holds
            also.

            If $f:\R \rightarrow \R$ is continuous at a point $x_0$, then for
            every $\epsilon>0$, there is a  $\delta>0$ such that
            $|f(x)-f(x_0)|<\epsilon$ whenever $0<|x-x_0|<\delta$. Then we notice
            that $x$ and  $ x_0$ are distinct, furthermore, $x_0-\delta<x
            <x_0+\delta$, hence $x \in (x_0-\delta,x_0+\delta)$ implies that
            $f(x) \in (f(x_0)-\epsilon,f(x_0)+\epsilon)$. Letting
            $V_{\delta}(x_0)=(x_0-\delta,x_0+\delta) $ and
            $V_{\epsilon}(f(x_0))=(f(x_0)-\epsilon,f(x_0)+\epsilon)$, we have
            that whenever  $x \in V_{\delta}(x_0)$, then $f(x) \in
            V_{\epsilon}(f(x_0)) \subseteq f^{-1}(V_{\delta}(x_0))$. And so the
            topological definition of continuity is equivialent to the real
            analytic definition of continuity.

        \item Let $f:\R \rightarrow \R_l$ be defined such that  $f(x)=x$ for all
            $x \in \R$. Take  $[a,b) \subseteq \R_l$, we have that
            $f^{-1}([a,b))=[a,b)$, which is not open in  $\R$  (under the
            standard topology), hence $f$ is not continuous. However, the map
            $g:\R_l \rightarrow \R$ defined the same way is continuous since
            $g^{-1}((a,b))$ is open in  $\R_l$.
    \end{enumerate}
\end{example} 

\begin{theorem}\label{1.7.1}
    Let $X$ and  $Y$ be topological spaces, and let  $f:X \rightarrow Y$ be a mapping of  $X$ into
    $Y$. Then the following are equivalent:
        \begin{enumerate}[label=(\arabic*)]
            \item $f$ is continuous.

            \item For every $A \subseteq X$,  $f(\bar{A}) \subseteq \bar{f(A)}$.

            \item For every closed set $B \subseteq Y$,  $f^{-1}(B)$ is closed in $X$.

            \item For each  $x \in X$ and each neighborhood  $V$ of  $f(x)$, tgere us a neighborhood
                $U$ of  $x$ such that  $f(U) \subseteq V$.
        \end{enumerate}
\end{theorem}
\begin{proof}
    Let $f$ be continuous and let  $A \subseteq X$. Consider the neighborhood  $V$ of  $f(x)$, then
    $f^{-1}(V)$ is open in $X$, and intersects  $A$ at a point  $y$. Then  $V \cap f(A)=f(y)$, thus 
    $f(x) \in \bar{f(A)}$.

    Now let $B$ be closed in  $Y$, and let  $A=f^{-1}(B)$. THen we have that $F(A)=f(f^{-1(B)})
    \subseteq B$, thus $x \in \bar{A}$.

    Now let $V$ be open in $Y$, so that  $B=\com{Y}{V}$ is closed in $Y$, and
    $f^{-1}(B)=\com{f^{-1}(Y)}{f^{-1}(V)}=\com{X}{f^-1}(V)$ which is closed in $X$, hence
    $f^{-1}(V)$ is open in $X$.

    Now let  $x \in X$, and let  $V$ be a neighborhood of  $f(x)$. Then $U=f^{-1}(V)$ is a
    neighborhood of $x$ for which  $f(U) \subseteq V$. Finally let $V$ be open in  $Y$, and let $x
    \in f^{-1}(V)$, then $f(x) \in V$, so  there is a neighborhood $U_x$ of  $x$ for which  $f(U_x) \subseteq V$,
    then $U_x \subseteq f^{-1}(V)$, then $f^{-1}(V)$ is a union of open sets, and hence open in $X$.
\end{proof}

\begin{definition}
    Let $X$ and  $Y$ be topological spaces, and  $f:X \rightarrow Y$ be a $1-1$ mapping of  $X$ onto
    $Y$. We call  $f$ a \textbf{homeomorphism} if both $f$ and  $f^{-1}$ are continuous.		
\end{definition}

\begin{lemma}\label{1.7.2} 
    Let $X$ and  $Y$ be topological spaces and let $f:X \rightarrow Y$ be a homeomorphism. Then
    $F(U)$ is open if and only if $U$ is open.
\end{lemma}
\begin{proof}
    We have that both $f:X \rightarrow Y$ and  $f^{-1}:Y \rightarrow X$ are contiuous $1-1$ of $X$
    and  $Y$ onto each other  (respectively). Now let $U$ be open in  $X$, then $U=f^{-1}(V)$, for 
    some set  $V$ open in $Y$. Notice then, that $f(U)=f(f^{-1}(V))=V$, thus $f(U)$ is open in $Y$. 
    Conversely, let $V=f(U)$ be open in $Y$ for some open set  $U$ in  $X$, then  $U=f^{-1}(V)$, so
    by definition of continuity, $U$  is open in $X$.
\end{proof}

\begin{definition}
    Let $X$ and  $Y$ be topological spaces and let  $f:X \rightarrow Y$ be a contiuous  $1-1$ maping
    of  $X$ into  $Y$, and consider  $f(X)$ as a subspace of $Y$. We call  $f:X \rightarrow f(X)$ a
    \textbf{topological imbedding} if $f$ is a homeomorphism of $X$ onto  $f(X)$.
\end{definition}

\begin{example}
    \begin{enumerate}[label=(\arabic*)]
        \item The map $f:\R \rightarrow \R$ defined by  $f(x)=3x+1$ is a homeomorphism whose inverse
            is $f^{-1}(y)-\frac{1}{3}(y-3)$, both $f$ and  $f^{-1}$ are contiuous.

        \item The map $f:(-1,1) \rightarrow \R$ defined by $f(x)=\frac{x^2}{1-x^2}$ has as its
            inverse the map $f:\R \rightarrow (-1,1)$ defined by
            $f^{-1}(y)=\frac{2y}{1+\sqrt{1+4y^2}}$. Both $f$ and  $f^{-1}$ are contiuous, so $f$ is
            a homeomorphism.

        \item Teh map  $g:\R_l \rightarrow \R$ defined by  $g(x)=x$ is not a homeomorphism, despite
            being contiuous, as $g^{-1}(1)$ is undefined.

        \item Let $S^1$ be the unit circle in $\R$, which is a subspace of $\R$, and define
            $f:[0,1) \rightarrow S^1$ by $f(t)=(\cos(t),\sin(t))$. Clearly $f$ is  $1-1$ onto
            $S^1$, and contiuous, however  $f^{-1}$ is not contiuous as $f([0,\frac{1}{4}))$ is not
            open in $S^1$ as  $f(0)$ is in no open set of $\R^2$ such that  $U \cap S^1=f([0,1))$.

        \item Consider the mappings $g:[0,1) \rightarrow \R^2$ by $f(t)=(\cos(2t\pi),\sin(2t\pi)))$.
            Now $g$ is  $1-1$ and continuous, and we have that  $g([0,1)) \subseteq S^1$, however
            since $g$ is not a homeomorphism,  $g$ fails to be a topological embedding.
    \end{enumerate}
\end{example} 
