%----------------------------------------------------------------------------------------
%	SECTION 1.1
%----------------------------------------------------------------------------------------

\section{Definitions and Examples}
\label{section1}

\begin{definition}
    We call a nonempty set $G$ a \textbf{group} under a binary operation $\cdot$ if the following
    hold:
        \begin{enumerate}[label=(\arabic*)]
            \item $a,b \in G$ implies  $a \cdot b \in G$.

            \item For all $a,b,c \in G$,  $(a \cdot b) \cdot c=a \cdot (b \cdot c)$.

            \item There is an element $e \in G$, called the \textbf{identity element} such that $a
                \cdot e=e \cdot a=a$, for all  $a \in G$.

            \item For all  $a \in G$, there is a corresponding element  $a^{-1}$, called the
            \textbf{inverse element} of $a$, such that  $a \cdot a^{-1}=a^{-1} \cdot a=e$.
        \end{enumerate}
        We call $G$ \textbf{abelian} (or \textbf{commutative}) if $a \cdot b = b \cdot a$, for all
        $a,b \in G$. We call  $|G|$ the \textbf{order} of $G$ and denote it  $\ord{G}$.
\end{definition}

\begin{example}
    \begin{enumerate}[label=(\arabic*)]
        \item Let $S$ be an  $n$ element set, and let  $S_n$ be the set of all  $1-1$ mappings of
            $S$ onto itself  (i.e all permutations of elements of $S$). Then $S_n$ forms a group
            over function composition $\circ$.

            Indeed, whenever  $f,g \in S_n$,  $f \circ g \in S_n$, likewise,  $f \circ (g \circ
            h)=(f \circ g) \circ h$. The identity map $i:S \rightarrow S$ defined by the rule  $i:s
            \rightarrow s$ serves as the identity element;  $f \circ i=i \circ f=f$. Finally since
            whenever $f \in G$,  $f$ is  $1-1$ andd onto,  $f^{-1}$ exists and is also $1-1$ and
            onto; moreover  $f \circ f^{-1}=f^{-1} \circ f=i$, so $f^{-1}$ is the inverse of $f$. It
            is also easy to see that  $\ord{S_n}=n!$. It is worth noting that $S_n$ is not ingeneral
            commutative, as  $f \circ g \neq g \circ f$.

        \item The integers $\Z$ form a group over  $+$  (the usual addition), but not over $\cdot$
            (the usual multiplication). The rationals $\Q$ do form a group under  $\cdot$. The reals
             $\R$ and the complex numbers $\C$ form abelian groups under both  $+$ and  $\cdot$.

         \item Let  $G=\{-1,1\}$m then $(G,\cdot)$ forms a group of order $2$, where  $\cdot$ is the
             usual multiplication.

         \item By example  $1$, we have that  $S_3$ forms a group of order  $3!=6$. Now consider the
             maps  $\phi:1 \rightarrow 2, 2 \rightarrow 3, 3 \rightarrow 3$ and $\psi:1 \rightarrow
             3, 2 \rightarrow 3, 3 \rightarrow 1$. We can check that  $\phi^2=\psi^3=i$, also notice
             that  $\phi\psi:1 \rightarrow 2, 2 \rightarrow 2, 3 \rightarrow 1$ and 
             $\psi\phi:1 \rightarrow 1, 2 \rightarrow 3, 3 \rightarrow 2$, so $\phi\psi \neq
             \psi\phi$. Likewise we also have  $\psi^2=\psi\psi: 1 \rightarrow 2, 2 \rightarrow 1, 3
             \rightarrow 2$ and $\psi^{-1}\phi: 1 \rightarrow 3, 2 \rightarrow 2, 3
             \rightarrow 1$. Indeed, in $ S_3$, $\phi\psi=\psi^{-1}\phi$; it turns out that $S_3$ is
             a special case of a more general group.

         \item  $\Z/n\Z$ forms an abelian group under  $+$  (addition $\mod{n}$), and that
             $U(\Z/n\Z)$ forms a group under $\cdot$  (multiplication $\mod{n}$).

         \item If we take $(G,\cdot)$ and $(H,*)$ to be groups, and consider their product $G \times
             H$, define the binary operation  $\times$ by taking  $(a,b) \times (c,d)=(a \cdot c,
             b*d)$, where $a,c \in G$ and  $b,d \in H$, then  $(g \times H, \times)$ forms a group.
    \end{enumerate}		
\end{example} 

\begin{definition}
    We say a group $G$ is \textbf{cyclic} if for some $g \in G$,  $G=\{g^i:i \in \Z\}$. We call $g$
    the \textbf {generator} of $G$ and write  $G=(g)$.
\end{definition}

\begin{lemma}\label{1.1.1}
    If $G$ is a group, then the following hold:
        \begin{enumerate}[label=(\arabic*)]
            \item The identity element is unique.

            \item Inverses are unique.

            \item  $(a^{-1})^{-1}=a$ for all $a \in G$.

            \item  $(ab)^{-1}=b^{-1}a^{-1}$.
        \end{enumerate}
\end{lemma}
\begin{proof}
    First suppose that $G$ has an additional identity element  $f$, that is for all $a \in G$,
    $af=fa=a$. Then we have that  (with $e$ the identity of  $G$), $ae=af$, then
    $(a^{-1}a)e=(a^{-1}a)f$, hence $e=f$.

    Now suppose that for some  $a \in G$, that  $a$ has an additional inverse element  $x$, then
    $ax=xa=e$, firthermore, since  $a^{-1}$ is the inverse of $a$, we have  $aa^{-1}=ax$, applying
    inverses again we get $(a^{-1}a)a^{-1}=(a^{-1}a)x$, hence $a^[-1]=x$.

    We have that $aa^{-1}=e$, and there exists a unique inverse element $(a^{-1})^{-1}$ of $a^{-1}$,
    hence $a(a^{-1}(a^{-1})^{-1})=(a^{-1})^{-1}$, hence we get that $a=(a^{-1})^{-1}$.

    Finally, we have that $ab(ab)^{-1}=e$, then $(a^{-1}a)b(ab)^{-1}=a^{-1}$, and so
    $b^{-1}b(ab)^{-1}=(ab)^{-1}=b^{-1}a^{-1}$.
\end{proof}

\begin{lemma}[The Cancelation laws]\label{1.1.2}
    Let $G$ be a group with  $a,b \in G$. Then the equations  $ax=b$ and  $ya=b$ have unique
    solutions. Moreover for $u,w \in G$,  $au=aw$ implies  $u=w$ and  $ua=wa$ implies  $u=w$.
\end{lemma}
\begin{proof}
    We have that $x=a^{-1}b$ and $y=ba^{-1}$ are the unique solutions to the equations. Now for $u,w
    \in G$, we have that  $au=aw$ has as solution  $u=(a^{-1}a)w=w$, the same holds for $ua=wa$.
\end{proof}
