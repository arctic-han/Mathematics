%----------------------------------------------------------------------------------------
%	SECTION 1.1
%----------------------------------------------------------------------------------------

\section{The Dihedral Group.}
\label{section1}

As we noted in a previous example, the group $S_3$ is a special case of a more broad group of
permutations. We can recall that  $\phi^2=\psi^3=i$, and that  $\phi\psi=\psi^{-1}\phi$, and indeed
$\ord{S_3}=6=3!=2(3)$. We would like to generalize this group structure further.

\begin{theorem}\label{1.1.2}
    Let $n \in \Z^+$ and let $D_{2n}$ be the set of all symmetries of a regular $n$-gon; that is the
    set of all permutation of points of the $n$-gon, defined by two maps $\tau:A \rightarrow -A$
    which is a transposition of opposite vertices, and $\rho:A \rightarrow A+1$ which is a rotation
    of the vertices about an angle of $\frac{2\pi}{n}$. Then $D_{2n}$ forms a group under function
    composition.
\end{theorem}
\begin{proof}
    Let $S$ be a regular  $n$-gon with vertices  $0,1, \dots, n$. Notice that  $\tau,\rho \in S_n$,
    so they are  $1-1$ maps of the  $n$-gon onto itself. By our definitions of  $\tau$ and  $\rho$,
    we have that  $\tau:i \rightarrow n-1$ and  $\rho:i \rightarrow i+1$. Hence  $\tau\rho:i
    \rightarrow n-i \rightarrow n-i+1$, which must coinide with some given vertex of $S$, hence
    $\tau\rho \in D_{2n}$; moreover, $D_{2n}$ inherits associativity from function composition.

    Now let $\iota:i \rightarrow i$ be the symmetry that leaves points unchange in $S$, clearly
    $\iota$ is the identity map, and so  $\tau\rho\itoa=\iota\tau\rho=\tau\rho$.

    Now how do we find the inverses? Notice that  $\tau:n-i \rightarrow i$, hence  $\tau^2:i
    \rightarrow n-i \rightarrow i$, that is $\tau^2=\iota$, and also notice that  $\rho^n:i
    \rightarrow i+1 \rightarrow i+2 \rightarrow \dots \rightarrow i+n=i$, so  $\rho^n=\iota$. This
    shows that  $\tau=\tau^{-1}$ and $\rho^{n-1}=\rho^{-1}$.  Then if $y \in D_{2n}$ such that
    $\tau\rho y=\iota$, then  $\rho y=\tau$, and  $y=\tau\rho^{n-1}$. Checking we get 
    that $\tau\rho(\tau\rho^{n-1})=(\tau\rho^{n-1})\tau\rho=\iota$. Therefore $D_{2n}$ is a group
    under $\circ$.
\end{proof}
\begin{corollary}
    $\orrd{D_{2n}}=2n$.
\end{corollary}
\begin{proof}
    We have that there are $n$ possible vertices to which  $i$ can mapped to via  $\rho$, so already
    there are  $n$ possible  $\rho$. Now we also have that  $\tau:i \rightarrow n-1$, which means
    that $i$ under $\tau$ can only be mapped to $n-1$. Since the elements of $D_{2n}$ are obviously
    of the form $\tau\rho^j$, for  $1 \leq j \leq n$, we see there are  $n$ possible  $\tau\rho^j$.
    Therefore, there are  $2n$ total symmetries of the  $n$-gon.
\end{proof}
\begin{remark}
    Now since $D_{2n}$ is obviously finite, ($\ord{D_{2n}}$ need not be known), then we can simply
    enumerate all the elements of $D_{2n}$, which are $D_{2n}=\{\iota, \tau, \rho, \rho^2, \dots.
    \rho^{n-1}, \tau\rho, \tau\rho^2, \dots, \tau\rho^{n-1}\}$. It is also worth noting that if
    $\tau\rho^i=\tau\rho^j$, then  $\rho^i=\rho^j$, hence  $i=j$, that is the elements of  $R_{2n}$
    are well defined.
\end{remark}
\begin{corollary}
    $\rho\tau=\tau\rho^{-1}$.		
\end{corollary}
\begin{proof}
    By direct computation, notice that $\rho\tau:i \rightarrow n-1 \rightarrow n-i+1$ and
    $\tau\rho^{-1}:i \rightarrow i-1 \rightarrow n-(i-1)=n-i+1$ (We can consider $\rho^{-1}$ also to
    be a rotation about the angle of $-\frac{2\pi}{n}$, hence it takes any vertex $i$ to  $i-1$).
    Hence $\rho\tau=\tau\rho^{-1}$.
\end{proof}
\begin{remark}
    This also shows that $\tau\rho \neq \rho\tau$, hence $D_{2n}$ is not commutative.
\end{remark}
\begin{corollary}
    For $i \in \Z^+$ with $1 \leq i \leq n$,  $\rho^i\tau=\tau\rho^{-i}$.
\end{corollary}
\begin{proof}
    Bu induction, the previous corollary gives $\rho^1\tau=\tau\rho^{-1}$. Now suppose that for all
    $1 \leq i \leq n$, that  $\rho^i\tau=\tau\rho^{-i}$, and consider $\rho^{i+1}$. If  $i+1=n$,
    then we are done, so take $i+1<n$. Then
    $\rho^{i+1}\tau=\rho(\rho^i\tau)=(\rho\tau)\rho^{-i}=\tau(\rho^{-1}\rho^{-i})=\tau\rho^{-i-1}$.
\end{proof}

\begin{figure}
    \centering
    \includegraphics[scale = 0.2]{Figures/Chapter1/D_14.eps}
    \caption{The dihedral group $D_{14}$ on $7$ points.}
    \label{fig_1.1}
\end{figure}

\begin{definition}
    We call the group $D_{2n}$ of symmetries of the regular $n$-gon the \textbf{dihedral group} of
    order $2n$.		
\end{definition}

Now we see that both $\tau$ and  $\rho$ generate  $D_{2n}$, furthermore, we have the relations
$\tau^2=\iota$ and  $\rho^n=\iota$ and  $\rho\tau=\tau\rho^{-1}$. This motivates the idea of
``generators'' of arbitrary groups.

\begin{definition}
    Let $G$ be a group with $S \subseteq G$. We call $S$ a set of \textbf{generators} of $G$ if
    every element in $G$ can be written as a finite product of elements of $S$ and their inverses.
    We say that $S$ \textbf{generates} $G$ and write $G=(s)$. We call a \textbf{relation} of the
    group any equation satisfied by the generators of $G$.
\end{definition}

\begin{example}
    \begin{enumerate}[label=(\arabic*)]
        \item If $G$ is any group, and  $S$ is the set of generators of  $G$ with  $|S|=1$, then
            $G$ is a cyclic group.

        \item In $(\Z,+)$, $\{1\}$ generates $\Z$, so  $\Z=\{1\}$.

        \item In $D_{2n}$, we have that $\{\tau,\rho\}$ generate all of $D_{2n}$, so
            $D_{2n}=(\tau,\rho)$. and $D_{2n}$ has the relations $\tau^2=\iota$,  $\rho^n=\iota$ and
             $\rho\tau=\tau\rho^{-1}$.
    \end{enumerate}		
\end{example} 

We can also write $G$ in terms of just its generators and relations as  $G=\{S:R_1, \dots, R_m\}$,
where $R_i$ is a relation of the generators for  $1 \leq i \leq m$. Then for the dihedral group, we
have  $D_{2n}=\{\tau,\rho:\tau^2=\rho^n=\iota, \text{ } \rho\tau=\tau\rho^{-1}\}$
