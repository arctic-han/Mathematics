%----------------------------------------------------------------------------------------
%	SECTION 1
%----------------------------------------------------------------------------------------

\section{Loops and Parallel elements.}

\begin{definition}
    We define a \textbf{loop} of a matroid to be an element $x \in S$ for which  $\{x\}$ is a
    dependent set. We say that two elements $x,y \in S$ are parallel if  $\{x,y\}$ is dependent, and
    they are both not loops.
\end{definition}

\begin{example}
    Let $M$ be the cycle matroid of the graph  $G$ in figure \ref{fig:1.3} has $2$ loops which we
    label $x_1$ and $ x_2$, and $4$ parallel elements which we label $y_1,y_2$ and $z_1,z_2$.

    \begin{figure} 
        \centering
        \includegraphics[scale = 0.1]{Figures/Chapter1/loopsAndParallels.png}
        \caption{The graph $G$ has $2$ loops and  $4$ parallel elements.}
        \fig{fig:1.3}
    \end{figure}
\end{example} 

We have the following proposition on loops.

\begin{proposition}\label{1.3.1}
    \begin{enumerate}[label=(\arabic*)]
        \item $x$ is a loop if and only if  $x \in \cl{\emptyset}$.

        \item If $x$ is a loop and $x \in A$, then  $A$ is dependent.

        \item If $x$ is a loop then  $x \in \cl{A}$ for all $A \subseteq S$.

        \item $x$ is a loop if and only if  $\{x\}$ is a circuit.

        \item $x$ is a loop if and only if it cannot be contained in any base.
    \end{enumerate}
\end{proposition}
\begin{proof}
    \begin{enumerate}[label=(\arabic*)]
        \item If $x$ is a loop, then  $\{x\}$ is dependent and since $\rank{\emptyset}=0$, we have
            $\rank{\emptyset \cup \{x\}}=\rank{\emptyset}=0$, so $x \in \cl{\emptyset}$. Conversely
            if $x \in \cl{\emptyset}$, then $\rank{\emptyset \cup \{x\}}=\rank{\emptyset}=0$, and
            since $x \in S$, $\{x\}$ is dependent.

        \item This is clear by definition and by $(1)$.

        \item If $x$ is a loop and if  $A$ were independent, then that would make  $\{x\}$ 
            independent; absurd.

        \item For $A \subseteq S$, since  $x$ is a loop,  $\rank{A \cup x}=\rank{A}$, so $x \in
            \cl{A}$.

        \item This is clear by definition.

        \item This follows from the contrapositive of proposition \ref{1.1.6}.
    \end{enumerate}
\end{proof}
\begin{corollary}
    A loop belongs to every flat.		
\end{corollary}
\begin{proof}
    If $x$ is a loop and $F$ is an arbitrary flat, then  $x \in \cl{\emptyset} \subseteq \cl{F}$, 
    and we also have that $x \in \cl{F}$, thus $x \in F$.
\end{proof}
\begin{corollary}
    $\emptyset$ is a flat if and only if  $M$ has no loops.
\end{corollary}
\begin{proof}
    If $\emptyset$ is a flat, then for every  $x \in S$,  $\rank{\emptyset \cup
    x}=\rank{\emptyset}+1$, thus there are no elements depending on $\emptyset$, thus there are no
    loops in $M$.

    Conversely if there are no loops in  $M$, and  $\emptyset$ is not a flat, then there is some
    element in $x \in S$ for which  $\rank{\emptyset \cup x}=\rank{\emptyset}$, thus $\{x\}$ is
    dependent, which makes $x$ a loop; absurd.
\end{proof}

And the following proposition for parallel elements.

\begin{proposition}\label{1.3.2}
    \begin{enumerate}[label=(\arabic*)]
        \item Distinct elements $x$ and  $y$ are parallel if and only if $\{x,y\}$ is a circuit.

        \item If $x$ is parallel to  $y$ and  $y$ is parallel to  $z$, then  $x$ is parallel to
            $z$.

        \item If $x \neq y$, $x$ is parallel to  $y$ if and only if  $x \in \cl{y}$ and $y \in
            \cl{x}$, and they are both not loops.

         \item If $x \in \cl{A}$ for $A \subseteq S$, and  $x$ is parallel to  $y$, then  $y \in \cl{A}$.

         \item If $A$ contains two parallel elements, then $A$ must be dependent.
    \end{enumerate}
\end{proposition}
\begin{proof}
    \begin{enumerate}[label=(\arabic*)]
        \item If $x=y$, then  $x$ is a loop. Now if  $x \neq y$ and  $x$ and  $y$ are parallel, then
            $\{x,y\}$ is dependent, and $\{x\}$ isn't a loop (hence it is independent), thus
            $\{x,y\}$ is minimally dependent. On the on the other hand, if $\{x,y\}$ is a circuit,
            then both $\{x\}$ and $\{y\}$ are independent, thus they are not loops, so $x$ and  $y$
            are parallel.

        \item If $x$ is parallel to  $y$, then  $\{x,y\}$ is dependent, and if $y$ and  $z$ are
            parallel, then  $\{y,z\}$ is also dependent. Then by the circuit elimination axiom,
            $\{x,y\}$ is dependent, and  $x$ and  $z$ are both not loops, hence they are parallel.

        \item If  $x$ is parallel to $y$, then  $\rank{\{x,y\}}=\rank{\{y\}}$ likewise
            $\rank{\{x,y\}=\rank{x}}$, thus $x \in \cl{\{y\}}$ and $y \in \cl{\{x\}}$; that they are
            both not parallel follows by definition.

        \item If $x$ depends on  $A$, and  $x$ is parallel to  $y$, then  $\rank{A \cup
            y}=\rank{A}$, hence $y$ also depends on $A$.

        \item This is obvious by the inheritance axiom.
    \end{enumerate}
\end{proof}
