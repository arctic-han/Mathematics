%----------------------------------------------------------------------------------------
%	SECTION 1
%----------------------------------------------------------------------------------------

\section{Examples}

We define some matroids, and observe the properties of a peculiar one.

\begin{proposition}\label{1.2.1}
    Let $S$ be a set with  $|S|=n$ and define  $\Ic=\{X \subseteq S: |X| \leq k\}$ for $k \leq n$.
    Then  $S$ is a matroid on  $S$.
\end{proposition}
\begin{proof}
    Clearly $\emptyset \in \Ic$, and if $X$ is independent, and  $Y \subseteq X$, then  $|Y| \leq
    |X| \leq k$, hence  $Y \in \Ic$. 

    Now take $Y,X \in \Ic$ with  $|Y|<|X|$. Now if  $|Y|+1=|X|=k$, the result is clear. Otherwise,
    choose an  $x \i \com{X}{Y}$, then since $|Y|<k$,  $|Y \cup x| \leq k$, and hence is
    independent.
\end{proof}

\begin{definition}
    We call the matroids on a set $S$, generated by the collection  $\Ic=\{X \subseteq S:|X| \leq
    \}$ the \textbf{uniform matroid} of rank $k$ on  $S$; and we denote it  $U_{n,k}$.
\end{definition}

We discuss some properties of the uniform matroid.

\begin{corollary}
    $\Bc(U_{n,k})=\{X \subseteq S: |X|=k\}$ and $\Cc(U_{n,k})=\{X \subseteq S: |X|=k+1\}$
\end{corollary}
\begin{proof}
    If $B \in \Ic(U_{n,k})$ is a base, then $|B| \leq k$ and by the maximality of $B$ for any $x \in
    \com{S}{B}$, $B \cup x \notin \Ic$, i.e.  $|B \cup x|=|B|+1>k$. It follows that  $|B|=k$.
    Likewise by the same reasoning we see by the minimality of any circuit $C \in \Cc$ that $|C|=k+1$
\end{proof}
\begin{remark}
    Since any base, and any dependent set in $U_{n,k}$ has size $\geq k$, it is easy to see that any
    set  $A$ with  $|A| \geq k$ is spanning in  $U_{n,k}$.
\end{remark}

\begin{corollary}
   For any $A \subseteq U_{n,k}$ $\rank{A}=\begin{cases}
                                        |A|, & \text{ if } |A| \leq k \\
                                        k, & \text{ if } |A|>k \\
                                    \end{cases}$
   and $\cl{A}=\begin{cases}
                    A, & \text{ if } |A|<k \\
                    S, & \text{ if } |A| \geq k \\
               \end{cases}$
\end{corollary}
\begin{proof}
    By definition, we have that $\rank{A}=\max\{|X|: X \subseteq A, |X|\leq k\}$. Now if $A$ is
    independent, then  $\rank{A}=A$. If $A$ is dependent, well since every dependent set of
    $U_{n,k}$ is spanning, choose a base $B \subseteq A$. Then  $\rank{A}=\rank{B}=k$.

    Now by the closure axioms,  $A \subseteq \cl{A} \subseteq S$. Suppose that $|A|<k$, and take  $x
    \in \cl{A}$, then $\rank{A \cup x}=\rank{A}+1=|A|+1$, hence $\rank{A}=|A|$, thus $x \in A$, so
$\cl{A}=A$. Now if $|A| \geq k$, then for any  $x \in \com{S}{A}$, $\rank{A \cup x}=\rank{A}+1 \geq
k+1>|A|+1$, this puts $x \in \cl{A}$, and bu consequence $S \subseteq \cl{A}$. Therefore $\cl{A}=S$.
\end{proof}
\begin{corollary}
    $\rank{U_{n,k}}=k$.
\end{corollary}

In the example 1.1.1 of section 1.1, we showed that a collection of linearlt independent subsets of
vectors of a vector space forms a matroid over that vector space.

\begin{definition}
    Let $V$ be a finite vector space, and let $M$ be the matroid on $V$ formed by taking all
    linearly independent subsets of vectors of $V$. We call a matroid isomorphic to $M$
    \textbf{vectorial}.
\end{definition}

\begin{proposition}\label{1.2.2}
    The rank of a vectorial matroid is the dimension of the vector space that it is isomorphic to.
\end{proposition}
\begin{proof}
    Since $\dim{V}=|B|$, where $B$ is a basis of linearly independent vectors of  $V$, it is easy to
    see that  $\rank{M}=\dim{V}$.
\end{proof}

\begin{definition}
    Let $G$ be a graph with edge set  $E$, and let  $X \in \Ic$ if and only if  $X$ contains no
    cycle of  $G$. Then  $\Ic$ is a collection of independent sets of a matroid, which we call the
    \textbf {cycle matroid} on $G$ and denote it  $M(G)$.		
\end{definition}

We defer the proof that the cycle is indeed a matroid for when we talk about matroids on graphs.

\begin{example}
    The complete graph $K_3$ has as it's cycle matroid the matroid  $U_{3,2}$ see figure
    \ref{fig:1.1}.
    \begin{figure}
        \centering
        \includegraphics[scale = 0.3]{Figures/Chapter1/k3cyclemat.png}
        \caption{The complete graph $K_3$.}
        \label{fig:1.1}
    \end{figure}
\end{example} 

We now talk about matroids arising from algebra. we give some proofs, but these matroids will be
discussed when approrpiate.

\begin{definition}
    Let $F$ be a field and  $K$ be an extension of $F$. We call a subset  $\{x_1, \dots, x_k\}
    \subseteq K$ of $K$ \textbf {algebraically dependent} if there is a polynomial $f$ with
    coefficients in  $F$ such that  $f(x_1, dots, x_k)=0$. Otherwise we say they are \textbf
    {algebraically independent}.
\end{definition}

\begin{theorem}\label{1.2.3}
    Let $F$ be a field,  $K$ an extension of  $F$, and let  $S \subseteq K$ be finite.  For any $X
    \subseteq S$, let  $X \in \Ic$ if and only if  $X$ is algebraically independent. Then  $\Ic$
    forms a matroid over  $S$..
\end{theorem}
\begin{proof}
    Clearly $\emptyset$ is algebraically independent. Now suppose that $X$
    is algebraically independent, and that  $Y \subeteq X$. Then for every polynomial 
    in $X$, $f \neq 0$. We have then there is a polyomial $g \in Y$ with $g \neq 0$ for which
     $f=g+h$ hence $Y$ is algebraically independent.

     Now suppose that $X=\{x_1, \dots, x_k\}$ and $Y=\{y_1, \dots, y_m\}$ are algebraically
     independent, with $m<k$, so that  $|Y|<|X|$. Now for every polynomial in $X$, $f$, $f(x_1,
     \dots, x_k) \neq 0$, and for every polynomial $g$ in  $Y$,  $g(y_1, \dots, y_m) \neq 0$. Now we
     can find a polynomial $ f_1$ in $X$ such that  $f=f_1(x_i)+f_2(x_1, \dots, x_{i-1},x_{i+1},
     \dots, x_k)$, where $ f_1(x_i) \neq 0$, for $x_i \in \com{X}{Y}$ (if $X \cap Y=\emptyset$, then
     choose any  $x_i$), then $f_1 \neq 0$ and $g \neq 0$ implies that  $f(x_i)+g(y_1, \dots, y_m)
     \neq 0$, then we see that $Y \cup x_i$ is also algebraically independent, whc=ich completes the
     proof.
\end{proof}

\begin{definition}
    Let $F$ be a field,  $K$ a field extension of  $F$, and  $S \subseteq K$ be finite. The
    collection of all algebraically independent subsets of  $S$ form a matroid on  $S$. We call
    these matroids \textbf {algebraic}.		
\end{definition}

\begin{definition}
    We call an element of Euclidean $d$-space, $x \in \R^d$ \textbf{affinely dependent} on a subset
    $\{x_1, \dots, x_r\} \subseteq \R^d$ if there exist real numbers $\lambda_i$ for  $1 \leq i \leq
    r$ such that:
        \begin{equation}
            \sum{\lambda_i}=1
        \end{equation}
    and
        \begin{equation}
            x=\sum{\lambda_ix_i}
        \end{equation} 
        We call a subset $X \subseteq \R^d$ \textbf {affinely independent} if no element $x \in X$ is
        affinely dependent on  $\com{X}{x}$.
\end{definition}

\begin{theorem}\label{1.2.4}
    Let $S \subseteq \R^d$. Then the collection of all affinely independent subsets of  $S$ form a
    matroid on $S$.
\end{theorem}
\begin{proof}
    Clearly $\emptyset$ is affinely indipendent, trivially. Let $X$ be affinely independent, and let
     $Y \subseteq X$. Then for  $x \in X$ and  $\{x_1, \dots, x_r\} \subseteq X$ with $x \neq x_i$
     for  $1 \leq i \leq r$, there are no real nubers  $\lambda_i$ for which  $\sum{\lambda_i}=1$ and
     $x=\sum{\lambda_ix_i}$. Now if $x \in Y$, take  $\{y_1, \dots, y_s\} \subseteq \{x_1, \dots,
     x_r\}$ and we are done. Now take $y \neq x \in Y$, and again take $\{y_1, \dots, y_s\} \subseteq \{x_1, \dots,
     x_r\}$ and where $y \neq y_j$ for  $1 \leq j \leq s$. Now if there are realnumbers  $\gamma_j$
     for which  $\sum{\gamma_j}=1$ and $y=\sum{\gamma_jy_j}$, then by inclusion, we can find
     addition $\gamma_i$ for which  $x=\sum{\gamma_ix_i}$, a contradiction; so $Y$ must also be
     affinely independent.
     
     Now let  $X$ and $Y$ be affinely independent with  $|Y|<|X|$, then for  $y \in Y$ and  $x \in
     X$ and  $\{x_1, \dots, x_r\}$, $\{y_1, \dots, y_s\}$ (not necessarily disjoint) with $x_i \neq
     x$ and  $y_j \neq y$ for  $1 \leq i \leq r$ and  $1 ]leq j \leq s$, there are no real numbers
     $\lambda_i$ and  $\gamma_j$ for which  $\sum{\lambda_i}=\sum{\gamma_j}=1$ and
     $x=\sum{\lambda_ix_i}$ and $y=\sum{\gamma_jy_j}$. Now if there is indeed a $\lambda_i$ and
     $x_i \notin \{y_1, \dots, y_s\}$ for which $y=\lambda_ix_i+\sum{\gamma_iy_i}$, that would imply
     that we can find real numbers $\lambda_i$ for which  $x=\sum{\lambda_i}$ which cannot happen.
     Thus $Y \cup x_i$ must be affinely independent.
\end{proof}

\begin{definition}
    Let $J$ be an abelian torsion group. An element  $g \in J$ is called \textbf {dependent} if for
    elements $ g_1, \dots, g_n \in J$ and integers $m \neq 0, k_1, \dots, k_m$ we have 
        \begin{equation}
            mg=k_1g_1+\dots+k_ng_n.
        \end{equation}
    A subset $Y \subseteq J$ is \textbf {independent} if there is no $y \in Y$ for which  $y$ is
    dependent on  $\com{Y}{y}$.
\end{definition}
We defer the proof of this theorem.
