%----------------------------------------------------------------------------------------
%	SECTION 1
%----------------------------------------------------------------------------------------

\section{Examples}

We define some matroids, and observe the properties of a peculiar one.

\begin{proposition}\label{1.2.1}
    Let $S$ be a set with  $|S|=n$ and define  $\Ic=\{X \subseteq S: |X| \leq k\}$ for $k \leq n$.
    Then  $S$ is a matroid on  $S$.
\end{proposition}
\begin{proof}
    Clearly $\emptyset \in \Ic$, and if $X$ is independent, and  $Y \subseteq X$, then  $|Y| \leq
    |X| \leq k$, hence  $Y \in \Ic$. 

    Now take $Y,X \in \Ic$ with  $|Y|<|X|$. Now if  $|Y|+1=|X|=k$, the result is clear. Otherwise,
    choose an  $x \i \com{X}{Y}$, then since $|Y|<k$,  $|Y \cup x| \leq k$, and hence is
    independent.
\end{proof}

\begin{definition}
    We call the matroids on a set $S$, generated by the collection  $\Ic=\{X \subseteq S:|X| \leq
    \}$ the \textbf{uniform matroid} of rank $k$ on  $S$; and we denote it  $U_{n,k}$.
\end{definition}

We discuss some properties of the uniform matroid.

\begin{corollary}
    $\Bc(U_{n,k})=\{X \subseteq S: |X|=k\}$ and $\Cc(U_{n,k})=\{X \subseteq S: |X|=k+1\}$
\end{corollary}
\begin{proof}
    If $B \in \Ic(U_{n,k})$ is a base, then $|B| \leq k$ and by the maximality of $B$ for any $x \in
    \com{S}{B}$, $B \cup x \notin \Ic$, i.e.  $|B \cup x|=|B|+1>k$. It follows that  $|B|=k$.
    Likewise by the same reasoning we see by the minimality of any circuit $C \in \Cc$ that $|C|=k+1$
\end{proof}
\begin{remark}
    Since any base, and any dependent set in $U_{n,k}$ has size $\geq k$, it is easy to see that any
    set  $A$ with  $|A| \geq k$ is spanning in  $U_{n,k}$.
\end{remark}

\begin{corollary}
   For any $A \subseteq U_{n,k}$ $\rank{A}=\begin{cases}
                                        |A|, & \text{ if } |A| \leq k \\
                                        k, & \text{ if } |A|>k \\
                                    \end{cases}$
   and $\cl{A}=\begin{cases}
                    A, & \text{ if } |A|<k \\
                    S, & \text{ if } |A| \geq k \\
               \end{cases}$
\end{corollary}
\begin{proof}
    By definition, we have that $\rank{A}=\max\{|X|: X \subseteq A, |X|\leq k\}$. Now if $A$ is
    independent, then  $\rank{A}=A$. If $A$ is dependent, well since every dependent set of
    $U_{n,k}$ is spanning, choose a base $B \subseteq A$. Then  $\rank{A}=\rank{B}=k$.

    Now by the closure axioms,  $A \subseteq \cl{A} \subseteq S$. Suppose that $|A|<k$, and take  $x
    \in \cl{A}$, then $\rank{A \cup x}=\rank{A}+1=|A|+1$, hence $\rank{A}=|A|$, thus $x \in A$, so
$\cl{A}=A$. Now if $|A| \geq k$, then for any  $x \in \com{S}{A}$, $\rank{A \cup x}=\rank{A}+1 \geq
k+1>|A|+1$, this puts $x \in \cl{A}$, and bu consequence $S \subseteq \cl{A}$. Therefore $\cl{A}=S$.
\end{proof}
\begin{corollary}
    $\rank{U_{n,k}}=k$.
\end{corollary}
