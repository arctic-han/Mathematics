%----------------------------------------------------------------------------------------
%	SECTION 1
%----------------------------------------------------------------------------------------

\section{Definitions}

We go over some fundamental definitions and theorems for matroids.

\begin{definition}[The Independence Axioms]
    We define a \textbf{matroid} $M$ to be a set $S$, called the \textbf {ground set}, together with
    a collection $\Ic \subseteq 2^S$ of subsets of  $S$ which we call \textbf {independent sets} to
    such that;
        \begin{enumerate}
            \item[(I1)] $\emptyset \in \Ic$.

            \item[(I2)] If $X \in \Ic$ and $Y \subseteq X$, then  $Y \in \Ic$. (Inheretence)

            \item[(I2)] If $X,Y \in \Ic$ such that  $|Y|<|X|$, then ther is an  $x \in \com{X}{Y}$ such
                that $Y \cup x \in \Ic$.  (Augmentation)
        \end{enumerate}
\end{definition}

There is one immediate example for a matroid.

\begin{example}
    Let $V$ be a finite vector space and let  $\Ic$ be the collection of all linearly independent
    subsets of vectors of  $V$. Clearly  $\emptyset \in V$, and if  $X$ is linearly independent, and
     $Y \subseteq X$, then  $Y$ is also linearly independent; so inheretence is satisfied.

     Now let $X$ be linearly independent, then  $\Span{S}$ must also be linearly independent. Now
     take $\beta \in \com{V}{\Span{S}}$ and for $\alpha_1, \dots, \alpha_m \in S$ take
        \begin{equation*}
            c_1\alpha_1+\dots+c_2\alpha_2+b\beta=0
        \end{equation*}
    If $b \neq 0$, then
        \begin{equation*}
            \beta=(-\frac{c_1}{b})\alpha_1+\dots+(-\frac{c_m}{b})\alpha_m
        \end{equation*}
    putting $\beta \in \Span{S}$, a contradiction, hence $b=0$, and so  $S \cup \beta$ is also
    linearly independent. Thus the augmentation axiom is satisfied and  $V$ is a matroid together
    with  $\Ic$.
\end{example} 

We define some additional concepts, all of which can be used in the definition of a matroid.

\begin{definition}
    A \textbf{base} of a matroid $M$ is a maximally independent subset of  $S$. We denote the
    collection of bases of $M$ by  $\Bc$. We say a subset of  $S$ is \textbf {spanning} in $M$ if it
    contains a base.
\end{definition}

\begin{definition}
    We define the \textbf{rank function} of a matroid to be the map $\rank:2^S \rightarrow \Z$
    defined by
        \begin{equation}
        \rank{A}=\max\{|X|:X \in \Ic \text{ and } X \subseteq A\}
        \end{equation} 
    We define the \textbf{rank} of the matroid to be $\rank{M}=\rank{S}$. We say $A \subseteq S$ is
    \textbf{closed}, or a \textbf{flat}, or a \textbf{subspace} of $M$ if for all  $x \in
    \com{S}{A}$, $\rank{A \cup x}=\rank{A}+1$, and we say $x$ \textbf{depends} on $A$.
\end{definition}

\begin{definition}
    We define the \textbf{closure operator} of a matroid to be the map $\cl:@^S \rightarrow 2^S$
    defined such that  $\cl{A}$is the set of all elements which depend on $A$; that is
        \begin{equation}
            \cl{A}=\{x \in \com{S}{A}:\rank{A \cup x}=\rank{A}+1\}.
        \end{equation}
\end{definition}

\begin{definition}
    A \textbf{dependent set} of a matroid is a subset $D \subseteq S$ which is not independent; this
    is $D \notin \Ic$. A \textbf {circuit} of a matroid is a minimally dependent set, and we denote
    the collection of all circuits as $\Cc$.
\end{definition}

Now one thing that makes matroids so interesting, is that they can be axiomatically defined in
various ways. We can not only define them in terms of independence, bu also in terms of bases,
circuits, rank, and closure. We give the theorems below that establish the axioms.

\begin{theorem}[The Base Axioms]\label{1.1.1}
    A nonempty collection $\Bc$ of subsesets of a set $S$ forms a set of bases of a matroid $M$ on
    $S$ if, and only if for:
        \begin{enumerate}
            \item[(B1)] $ B_1,B_2 \in \Bc$, and $x \in \com{B_1}{B_2}$, there is a $y \in
                \com{B_2}{B_1}$ such that $\com{(B_1 \cup y)}{x} \in \Bc$.
        \end{enumerate}
\end{theorem}

\begin{theorem}[The Circuit Axioms]\label{1.1.2}
    A nonempty collection $\Cc$ of subsesets of a set $S$ forms a set of circuits of a matroid $M$ on
    $S$ if, and only if:
        \begin{enumerate}
            \item[(C1)] If $Y \in \Cc$ and  $X \neq Y$, then  $X \not\subseteq Y$.

            \item[(C2)] If $ C_1,C_2 \in \Cc$ are distinct, and $z \in C_1 \cap C_2$, then there is
                a circuit $ C_3 \in \Cc$ such that $ C_3 \subseteq \com{(C_1\cup C_2)}{z}$.
        \end{enumerate}
\end{theorem}

\begin{theorem}[The First Rank Axioms]\label{1.1.3}
    Let $S$ be a set. A map $\rank:2^S \rightarrow \Z$ is the rank function of a matroid on $S$ if
    and only if for $X \subseteq S$ and  $y,z \in S$
        \begin{enumerate}
            \item[(R1)] $\rank{\emptyset}=1$.

            \item[(R2)] $\rank{X} \leq \rank{X \cup y} \leq \rank{X}+1$.

            \item[(R3)] If $\rank{X \cup y}=\rank{X \cup z}=\rank{X}$, then $\ranl{x \cup y \cup
                z}=\rank{X}$.
        \end{enumerate}
\end{theorem}

\begin{theorem}[The Second Rank Axioms]\label{1.1.4}
    Let $S$ be a set. A map $\rank:2^S \rightarrow \Z$ is the rank function of a matroid on $S$ if
    and only if for $X,Y \subseteq S$
        \begin{enumerate}
            \item[(R$'$1)] $0 \leq \rank{X} \leq |X|$.

            \item[(R$'$2)] If $X \subet Y$, then  $\rank{X} \leq \rank{Y}$ 

            \item[(R\`3)] $\rank{X \cup Y}+\rank{X \cap Y} \leq \rank{X}+\rank{Y}$.
        \end{enumerate}
\end{theorem}

\begin{theorem}[The Closure Axioms]\label{1.1.5}
    Let $S$ be a set. A map $\cl:2^S \rightarrow 2^S$ is the closure operator of a matroid on $S$ if
    and only if for $X,Y \subseteq S$, and $x,y \in S$
        \begin{enumerate}
            \item[(S1)] $X \subseteq \cl{X}$.

            \item[(S2)] If $Y \subseteq X$, then  $\cl{Y} \subseteq \cl{X}$.

            \item[(S3)] $\cl{X}=\cl{(\cl{X})}$.

            \item If $y \notin \cl{X}$ and $y \in \cl{(X \cup x)}$, then $x \in \cl{(X \\cup y)}$.
        \end{enumerate}
\end{theorem}

We defer their proofs to the relevant sections.

We can already provea fact about matroids.

\begin{proposition}\label{1.1.6}
    If an element of a matroid belongs to every base, then it can belong to no circuit of the
    matroid.
\end{proposition}
\begin{proof}
    Let $M$ be a matroid and let  $\Bc$ be the collection of all bases of $M$, $\Cc$ the collection
    of all circuits of $M$, and  $\Ic$ the collection of all independent sets of  $M$. Take  $x \in
    X=\bigcap_{B \in \Bc}{B}$ and suppose that $x \in C$ for $C \in \Cc$. By theorem \ref{1.1.1}, we
    have that $X \neq \emptyset$, moreover since  $x \in C$,  $C \subseteq X$, i.e  $C \in \Bc$. mow
    notice that since  $C$ is a circuit, it is dependent, so  $C \notin \Ic$, but we have that  $C
    \in \Bc$ which makes it a base, and hence independent; so  $C \in \Ic$, a contradiction.
\end{proof}

\begin{definition}
    We say that two matroids $ M_1$ and $ M_2$ on $ S_1$ and $ S_2$ respectively are
    \textbf{isomorphic} if there is a $1-1$ map  $\phi:S_1 \rightarrow S_2$ of $S_1$ onto $ S_2$ 
    such that if $X \subseteq S_1$ is independent in $ M_1$, then $\phi(X) \subseteq S_2$ is
    independent in $ M_2$. We write $ M_1 \simeq M_2$ to denote isomorphism. 		
\end{definition}

\begin{example}
    We list all nonisomorphic matroids on a set of $n$ elements for  $n=1,2,3$.		
        \begin{enumerate}
            \item[\underline{$n=1$}] For $S=\{1\}$, we have $ M_1=\emptyset$ and $ M_2=2^S$. There
                are $2^1=2$ matroids on  $S$.

            \item [\underline{$n=2$}] On $S=\{1,2\}$we have
                \begin{align*}
                    M_1 &= \emptyset \\
                    M_2 &= \{\emptyset, \{1\}\} \\
                    M_3 &= \{\emptyset \{1\},\{2\}\} \\
                    M_4 &= \{\emptyset,\{1\},\{2\},\{1,2\}\}=2^S
                \end{align*}
                there are $2^2=4$ matroids on $S$ 

            \item [\underline{$n=3$}] For $S=\{1,2,3\}$ we have
                \begin{align*}
                    M_1 &= \emptyset \\
                    M_2 &= \{\emptyset, \{1\}\} \\
                    M_3 &= \{\emptyset \{1\},\{2\}\} \\
                    M_4 &= \{\emptyset,\{1\},\{2\},\{1,2\}\} \\
                    M_5 &= \{\emptyset, \{1\}, \{2\}, \{3\}\} \\
                    M_6 &= \{\emptyset, \{1\}, \{2\}, \{3\}, \{1,2\}, \{1,3\}\} \\
                    M_7 &= \{\emptyset, \{1\}, \{2\}, \{3\}, \{1,2\}, \{1,3\}, \{2,3\}\} \\
                    M_8 &= \{\emptyset, \{1\}, \{2\}, \{3\}, \{1,2\}, \{1,3\}, \{2,3\},
                    \{1,2,3\}\}=2^S \\
                \end{align*}
                There are $2^3=8$ matroids on  $S$.
        \end{enumerate}

        Now one might be tempted to generalize that there are a total of $2^n$ matroids on a given
        $n$ element set, however a quick check for  $n=4$ concludes that that is not the case.

        \begin{enumerate}
            \item [\underline{$n=4$}] For $S=\{1,2,3,4\}$ we have
                \begin{align*}
                    M_1 &= \emptyset \\
                    M_2 &= \{\emptyset, \{1\}\} \\
                    M_3 &= \{\emptyset \{1\},\{2\}\} \\
                    M_4 &= \{\emptyset,\{1\},\{2\},\{1,2\}\} \\
                    M_5 &= \{\emptyset, \{1\}, \{2\}, \{3\}\} \\
                    M_6 &= \{\emptyset, \{1\}, \{2\}, \{3\}, \{1,2\}, \{1,3\}\} \\
                    M_7 &= \{\emptyset, \{1\}, \{2\}, \{3\}, \{1,2\}, \{1,3\}, \{2,3\}\} \\
                    M_8 &= \{\emptyset, \{1\}, \{2\}, \{3\}, \{1,2\}, \{1,3\}, \{2,3\},
                    \{1,2,3\}\} \\
                    M_9 &= \{\emptyset, \{1\}, \{2\}, \{3\}, \{4\}\} \\
                    M_{10} &= \{\emptyset, \{1\}, \{2\}, \{3\}, \{4\}, \{1,2\}, \{1,3\}, \{1,4\}\}
                    \\
                    M_{11} &= \{\emptyset, \{1\}, \{2\}, \{3\}, \{4\}, \{1,2\}, \{1,3\}, \{1,4\},
                    \{2,3\}\} \\
                    M_{12} &= \{\emptyset, \{1\}, \{2\}, \{3\}, \{4\}, \{1,2\}, \{1,3\}, \{1,4\},
                    \{2,3\}, \{2,4\}, \{3,4\}\} \\
                    M_{13} &= \{\emptyset, \{1\}, \{2\}, \{3\}, \{4\}, \{1,2\}, \{1,3\}, \{1,4\},
                    \{2,3\}, \{2,4\}, \{3,4\}, \{1,2,3\}\} \\
                    M_{14} &= \{\emptyset, \{1\}, \{2\}, \{3\}, \{4\}, \{1,2\}, \{1,3\}, \{1,4\},
                    \{2,3\}, \{2,4\}, \{3,4\}, \{1,2,3\}, \{1,2,4\}\} \\
                    M_{15} &= \{\emptyset, \{1\}, \{2\}, \{3\}, \{4\}, \{1,2\}, \{1,3\}, \{1,4\},
                    \{2,3\}, \{2,4\}, \{3,4\}, \{1,2,3\}, \{1,2,4\} \{1,3,4\}\} \\
                    M_{16} &= \{\emptyset, \{1\}, \{2\}, \{3\}, \{4\}, \{1,2\}, \{1,3\}, \{1,4\},
                    \{2,3\}, \{2,4\}, \{3,4\}, \{1,2,3\}, \{1,2,4\} \{1,3,4\}, \\ & \{2,3,4\}\} \\
                    M_{17} &= \{\emptyset, \{1\}, \{2\}, \{3\}, \{4\}, \{1,2\}, \{1,3\}, \{1,4\},
                    \{2,3\}, \{2,4\}, \{3,4\}, \{1,2,3\}, \{1,2,4\} \{1,3,4\}, \\ & \{2,3,4\},
                        \{1,2,3,4\}\}=2^S 
                \end{align*}
        \end{enumerate}
        Notice that $S=\{1,2,3,4\}$ has $17=2^4+1$ matroids.
\end{example} 
