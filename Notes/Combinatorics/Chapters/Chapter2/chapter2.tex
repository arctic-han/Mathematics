% Chapter Template

\chapter{Binomial Coefficients} % Main chapter title

\label{Chapter 2} % Change X to a consecutive number; for referencing this chapter elsewhere, use \ref{ChapterX}

%% to include section files use the \input{} command.

\begin{definition}
    Let $n \in \N$ and  $k \in \Z$. We define the \textbf {binomial coefficient} of $n$ \textbf
    {choose} $k$ to be 
        \begin{equation}
        \binom{n}{k}=\begin{cases}
            \frac{n!}{k!(n-k)!} & \text{, if } 0 \leq k \leq n \\
            0                   & \text{, if } k>n \text{ or } k<0 \\
                    \end{cases}
        \end{equation} 
    and where $\binom{n}{0}=\binom{n}{n}=1$.
\end{definition}

The solution to problem $3$ is a sufficient proof for the relation; and the fact that
$\binom{n}{k}=0$ for $n<k$ and  $k<0$ is evident since there can be no  $k$ element subsets of an
$n$ element set under those conditions. What follows are some fundamental lemmas about the binomial
coefficient.

\begin{lemma}[Symmetry]\label{2.0.1}
    $\binom{n}{k}=\binom{n}{n-k}$
\end{lemma}
\begin{proof}
    $\binom{n}{k}=\frac{n!}{k!(n-k)!}=\frac{n!}{(n-k)!(n-(n-k))!}=\binom{n}{n-k}$.
\end{proof}

\begin{lemma}[Pascal's Lemma]\label{2.0.2}
    If $n \in \N$ and  $k \in Z$ then 
        \begin{equation}
            \binom{n}{k}=\binom{n-1}{k}+\binom{n-1}{k-1}.
        \end{equation} 
\end{lemma}
\begin{proof}
    For $k<0$ or  $k>n$, the result is obvious by definition. Now suppose that $0 \leq k \leq n$,
    then
    $\binom{n-1}{k}+\binom{n-1}{k-1}=\frac{(n-1)!}{k!(n-k-1)!}+\frac{(n-1)!}{(k-1)!(n-k)!}=(n-1)!\frac{(n-k)+k}{k!(n-k)!}=\binom{n}{k}$.
\end{proof}
\begin{remark} 
    We can use Pascal's lemma to construct Pascal's triangle in figure \ref{fig:2.1}.
\end{remark}

\begin{figure} 
    \centering
        \begin{tabular}{>{$n=}l<{$\hspace{12pt}$}*{13}{c}}
            &&&&&&&1&&&&&&\\
            &&&&&&1&&1&&&&&\\
            &&&&&1&&2&&1&&&&\\
            &&&&1&&3&&3&&1&&&\\
            &&&1&&4&&6&&4&&1&&\\
            &&1&&5&&10&&10&&5&&1&\\
            &1&&6&&15&&20&&15&&6&&1
        \end{tabular}
        \caption{Pascal's triangle up to $n=6$.}
        \label{fig:2.1}
\end{figure}

\begin{theorem}[The Binomial Theorem]\label{2.0.3}
    If $n \in \N$, then
        \begin{equation}
            (x+y)^n=\sum_{k}{\binom{n}{k}x^ky^{n-k}}
        \end{equation} 
\end{theorem}
\begin{proof}
    We use induction on $n$. Notice that  $(x+y)^0=1=\binom{0}{0}x^0y^0$, and
    $(x+y)^1=x+y=\binom{1}{0}x^0y^1+\binom{1}{1}x^1y^0$
\end{proof}
\begin{proof}
    We use induction on $n$. Notice that  $(x+y)^0=\binom{0}{0}x^0y^0$ and
    $(x+y)^1=x+y=\binom{1}{0}x^0y^1+\binom{1}{1}x^1y^0$. Now suppose that the theorem holds for $n
    \geq 1$. Then
    $(x+y)^{n+1}=(x+y)\sum{\binom{n}{k}x^ny^{n-k}}=
    \sum{\binom{n}{k}x^{k+1}y^{n-k}}+\sum{\binom{n}{k}x^ky^{n+1-k}}=\sum{(\binom{n}{k-1}\binom{n}{k})x^ky^{n+1-k}}=
    \sum{\binom{n+1}{k}x^ky^{n+1-k}}$, by Pascal's lemma.
\end{proof}
\begin{corollary}
    $\sum_k{\binom{n}{k}}=2^n$ and there are exactly $2^n$ subsets of an  $n$ element set.		
\end{corollary}
\begin{proof}
    Exapand $(1+1)^n$, also notice that since there are $\binom{n}{k}$ possible $k$ element subsets
    of a given  $n$ element set, then  all we need to do is  take the sum of $\binom{n}{k}$ over
    $k$.
\end{proof}
\begin{corollary}
    $\sum_{k}{(-1)^k\binom{n}{k}}=\begin{cases}
                                    0 & , \text{ if } n \geq 1
                                    1 & , \text{ if } n=0
                                \end{cases}$
\end{corollary}
\begin{proof}
    Expand $(-1+1)^n$.
\end{proof}

\begin{lemma}\label{2.0.4}
    If $m,n \in \N$, then  $\sum_{k}{\binom{k}{m}}=\binom{n+1}{m+1}$.
\end{lemma}
\begin{proof}
    For $n=0$, the result follows,  $\binom{0}{0}=1=\binom{1}{1}$, and if $m>0$, then each side is
    $0$. Now suppose for $n \geq 0$ that the lemma holds. Then
    $\sum_{k=0}^{n+1}{\binom{k}{m}}=\binom{n+1}{m}+\binom{n+1}{m+1}=\binom{n+1}{m+1}$ by Pascal's
    lemma.
\end{proof}

\begin{theorem}[Vandermonde's Convolution]\label{2.0.5}
    If $m,n \in \N$ and  $l,p \in \Z$, then
        \begin{equation}
            \sum_{k}{\binom{m}{p+k}\binom{n}{l-k}}=\binom{m+n}{l+p}
        \end{equation} 
\end{theorem}
\begin{proof}
    By induction on $n$, for  $n=0$,  $\binom{m}{l+p}=\sum{\binom{m}{p+k}}$. Now suppose that the
    theorem holds for $n \geq 0$, then by Pascal's lemma we have that
    $\binom{m+n+1}{p+l}=\binom{m+n}{p+l}+\binom{m+n}{p+l-1}=\sum{\binom{m}{p+k}\binom{n}{l-k}}=
    \binom{m}{p+k}\binom{n+1}{0}+\sum{\binom{m}{p+k}\binom{n+1}{l-k}}=\sum{\binom{m}{p+k}\binom{n+1}{l-k}}$.
\end{proof}
