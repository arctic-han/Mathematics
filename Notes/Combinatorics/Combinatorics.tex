\documentclass[12pt]{book}

\usepackage[margin=1in]{geometry}
\usepackage{amsmath,amsfonts,amsthm,amssymb,graphicx,mathtools,tikz,hyperref}
\usetikzlibrary{positioning}
\input{Library/Fonts/BlackboardBold}
\input{Library/Fonts/MathCalligraphy}
\newcommand{\ita}[1]{\textit{#1}}
\newcommand{\com}[2]{#1\backslash#2}
\newcommand\idea[1]{\begin{gather*}#1\end{gather*}}
\newcommand\ef{\ita{f} }
\newcommand\eff{\ita{f}}
\newcommand\proofs[1]{\begin{proof}#1\end{proof}}
\newcommand\inv[1]{#1^{-1}}
\newcommand\setb[1]{\{#1\}}
\newcommand\en{\ita{n }}
\newcommand{\vbrack}[1]{\langle #1\rangle}
\newcommand{\bar}[1]{\overline{#1}}
\newcommand{\bbar}[1]{\underline{#1}}

\theoremstyle{plain}
\newtheorem{axiom}{Axiom}[section]
\newtheorem{theorem}{Theorem}[section]
\newtheorem{lemma}[theorem]{Lemma}
\newtheorem{proposition}[theorem]{Proposition}
\newtheorem*{corollary}{Corollary}

\theoremstyle{definition}
\newtheorem{definition}{Definition}[section]
\newtheorem{conjecture}{Conjecture}[section]
\newtheorem{example}{Example}[section]
\newtheorem{problem}{Problem}
\newtheorem*{solution}{Solution}

\theoremstyle{remark}
\newtheorem*{remark}{Remark}
\newtheorem{claim}{Claim}
\newtheorem*{note}{Note}

\title{Notes on Combinatorics}
\author{Alec Zabel-Mena \\ Text -}
\date{\today}

\begin{document}

\maketitle

% Chapter Template

\chapter{Fundamental of Matroid Theory.} % Main chapter title

\label{Chapter 1} % Change X to a consecutive number; for referencing this chapter elsewhere, use \ref{ChapterX}

%% to include section files use the \input{} command.

%----------------------------------------------------------------------------------------
%	SECTION 1.1
%----------------------------------------------------------------------------------------

\section{The Product Topology.}

We now explore more about the product topology.

\begin{definition}
    LEt $J$ be an indexed set, and let  $X$ be a set. We define a  \textbf{$J$-tuple} of elements of
    $X$ to be a map $x:J \rightarrow X$, where if  $\alpha \in J$, then  $x(\alpha)=x_{\alpha}$, and
    we call it the \textbf{$\alpha$-th coordinate} of $x$. We write  $(x_{\alpha})_\alpha \in J$, or
just simply $(x_{\alpha})$
\end{definition}

\begin{definition}
Let $\{A_{\alpha}\}$ be an indexed family, and let $X=\bigcup_{\alpha \in J}{A_{\alpha}}$. We define
the \textbf{cartesian product} of $\{A_{\alpha}\}$, $\prod_{\alpha \in J}{A_\alpha}$ to be the set of
all $J$-tuples  $(x_{\alpha})$ of elements of $X$, where  $x_{\alpha} \in A_{\alpha}$
\end{definition}

\begin{theorem}\label{2.1.1}
    Let $\{X_{\alpha}\}$ be a family of topological spaces, and consider the cartesian product
    $\prod{X_{\alpha}}$. Then the collection of all cartesian products $\prod{U_{\alpha}}$, where
    $U_{\alpha}$ is open in $X_{\alpha}$, for all $\alpha$, forms a basis for the topology on
    $\prod{X_{\alpha}}$.
\end{theorem}
\begin{proof}
    Clearly $\prod{X_{\alpha}}$ itself is a basis element by the first condition. Now consider
    $\prod{U_{\alpha}}$ and $\prod{V_{\alpha}}$, then $\prod{U_{\alpha}} \cap
    \prod{V_{\alpha}}=\prod{U_{\alpha} \cap V_{\apha}}$, which is also a basis element.
\end{proof}

\begin{definition}
Let $\{X_{\alpha}$ be a family of topological spaces, and take as basis the collection of all
    products $\prod{U_{\alpha}}$ where $U_{\alpha}$, where $U_{\alpha}$ is open in $X_{\alpha}$.
    We call the topology generated by this basis the \textbf{box topology} on $\prod{X_{\alpha}}$.		
\end{definition}

\begin{definition}
    Let $\pi_{\beta}:\prod{X_{\alpha}} \rightarrow X_{\beta}$ be defined by
    $\pi_{\beta}((x_\alpha))=x_\beta$. We call this map the \textbf{projection mapping} of
    $\prod{X_{\alpha}}$ onto $X_{\beta}$		
\end{definition}

\begin{theorem}\label{2.1.2}
    Let $\Sc_{\beta}=\{\pi^{-1}_{\beta}(U_{\beta}):U_{\beta} \text{ is open in } X_{\beta}\}$, and let
    $\Sc=\bigcup{\Sc_{\beta}}$. Then $\Sc$ forms the basis for a topology on  $\prod{X_[\alpha]}$.
\end{theorem}
\begin{proof}
    Since $U_{\beta}$ is open in $X_{\beta}$, $\pi^{-1}_{\beta}(U_{\beta})) \subseteq
    \prod{X_\alpha}$. Taking $\bigcup{\Sc}$, we get that $\bigcup{\pi^{-1}_{\beta}(U_{\beta}))} =
    \prod{X_\{beta}$ for all $\beta$. Thus  $\Sc$ is a subbasis.
\end{proof}


\begin{definition}
    Let $\pi_{\beta}$ be a projection mapping of $\prod{X_{\alpha}}$ onto $X_[\beta]$, and take as
    subbasis the collection of all $\pi^{-1}_{\beta}(U_{\beta})$ where $U_{\beta}$ is open in
    $X_{\beta}$. We call the topology generated by this subbasis the \textbf{product space
    topology}, or more generally the \textbf{product topology} on $\prod{X_{\alpha}}$.		
\end{definition}

%----------------------------------------------------------------------------------------
%	SECTION 1.1
%----------------------------------------------------------------------------------------

\section{The Dihedral Group.}
\label{section1}

As we noted in a previous example, the group $S_3$ is a special case of a more broad group of
permutations. We can recall that  $\phi^2=\psi^3=i$, and that  $\phi\psi=\psi^{-1}\phi$, and indeed
$\ord{S_3}=6=3!=2(3)$. We would like to generalize this group structure further.

\begin{theorem}\label{1.1.2}
    Let $n \in \Z^+$ and let $D_{2n}$ be the set of all symmetries of a regular $n$-gon; that is the
    set of all permutation of points of the $n$-gon, defined by two maps $\tau:A \rightarrow -A$
    which is a transposition of opposite vertices, and $\rho:A \rightarrow A+1$ which is a rotation
    of the vertices about an angle of $\frac{2\pi}{n}$. Then $D_{2n}$ forms a group under function
    composition.
\end{theorem}
\begin{proof}
    Let $S$ be a regular  $n$-gon with vertices  $0,1, \dots, n$. Notice that  $\tau,\rho \in S_n$,
    so they are  $1-1$ maps of the  $n$-gon onto itself. By our definitions of  $\tau$ and  $\rho$,
    we have that  $\tau:i \rightarrow n-1$ and  $\rho:i \rightarrow i+1$. Hence  $\tau\rho:i
    \rightarrow n-i \rightarrow n-i+1$, which must coinide with some given vertex of $S$, hence
    $\tau\rho \in D_{2n}$; moreover, $D_{2n}$ inherits associativity from function composition.

    Now let $\iota:i \rightarrow i$ be the symmetry that leaves points unchange in $S$, clearly
    $\iota$ is the identity map, and so  $\tau\rho\itoa=\iota\tau\rho=\tau\rho$.

    Now how do we find the inverses? Notice that  $\tau:n-i \rightarrow i$, hence  $\tau^2:i
    \rightarrow n-i \rightarrow i$, that is $\tau^2=\iota$, and also notice that  $\rho^n:i
    \rightarrow i+1 \rightarrow i+2 \rightarrow \dots \rightarrow i+n=i$, so  $\rho^n=\iota$. This
    shows that  $\tau=\tau^{-1}$ and $\rho^{n-1}=\rho^{-1}$.  Then if $y \in D_{2n}$ such that
    $\tau\rho y=\iota$, then  $\rho y=\tau$, and  $y=\tau\rho^{n-1}$. Checking we get 
    that $\tau\rho(\tau\rho^{n-1})=(\tau\rho^{n-1})\tau\rho=\iota$. Therefore $D_{2n}$ is a group
    under $\circ$.
\end{proof}
\begin{corollary}
    $\orrd{D_{2n}}=2n$.
\end{corollary}
\begin{proof}
    We have that there are $n$ possible vertices to which  $i$ can mapped to via  $\rho$, so already
    there are  $n$ possible  $\rho$. Now we also have that  $\tau:i \rightarrow n-1$, which means
    that $i$ under $\tau$ can only be mapped to $n-1$. Since the elements of $D_{2n}$ are obviously
    of the form $\tau\rho^j$, for  $1 \leq j \leq n$, we see there are  $n$ possible  $\tau\rho^j$.
    Therefore, there are  $2n$ total symmetries of the  $n$-gon.
\end{proof}
\begin{remark}
    Now since $D_{2n}$ is obviously finite, ($\ord{D_{2n}}$ need not be known), then we can simply
    enumerate all the elements of $D_{2n}$, which are $D_{2n}=\{\iota, \tau, \rho, \rho^2, \dots.
    \rho^{n-1}, \tau\rho, \tau\rho^2, \dots, \tau\rho^{n-1}\}$. It is also worth noting that if
    $\tau\rho^i=\tau\rho^j$, then  $\rho^i=\rho^j$, hence  $i=j$, that is the elements of  $R_{2n}$
    are well defined.
\end{remark}
\begin{corollary}
    $\rho\tau=\tau\rho^{-1}$.		
\end{corollary}
\begin{proof}
    By direct computation, notice that $\rho\tau:i \rightarrow n-1 \rightarrow n-i+1$ and
    $\tau\rho^{-1}:i \rightarrow i-1 \rightarrow n-(i-1)=n-i+1$ (We can consider $\rho^{-1}$ also to
    be a rotation about the angle of $-\frac{2\pi}{n}$, hence it takes any vertex $i$ to  $i-1$).
    Hence $\rho\tau=\tau\rho^{-1}$.
\end{proof}
\begin{remark}
    This also shows that $\tau\rho \neq \rho\tau$, hence $D_{2n}$ is not commutative.
\end{remark}
\begin{corollary}
    For $i \in \Z^+$ with $1 \leq i \leq n$,  $\rho^i\tau=\tau\rho^{-i}$.
\end{corollary}
\begin{proof}
    Bu induction, the previous corollary gives $\rho^1\tau=\tau\rho^{-1}$. Now suppose that for all
    $1 \leq i \leq n$, that  $\rho^i\tau=\tau\rho^{-i}$, and consider $\rho^{i+1}$. If  $i+1=n$,
    then we are done, so take $i+1<n$. Then
    $\rho^{i+1}\tau=\rho(\rho^i\tau)=(\rho\tau)\rho^{-i}=\tau(\rho^{-1}\rho^{-i})=\tau\rho^{-i-1}$.
\end{proof}

\begin{figure}
    \centering
    \includegraphics[scale = 0.2]{Figures/Chapter1/D_14.eps}
    \caption{The dihedral group $D_{14}$ on $7$ points.}
    \label{fig_1.1}
\end{figure}

\begin{definition}
    We call the group $D_{2n}$ of symmetries of the regular $n$-gon the \textbf{dihedral group} of
    order $2n$.		
\end{definition}

Now we see that both $\tau$ and  $\rho$ generate  $D_{2n}$, furthermore, we have the relations
$\tau^2=\iota$ and  $\rho^n=\iota$ and  $\rho\tau=\tau\rho^{-1}$. This motivates the idea of
``generators'' of arbitrary groups.

\begin{definition}
    Let $G$ be a group with $S \subseteq G$. We call $S$ a set of \textbf{generators} of $G$ if
    every element in $G$ can be written as a finite product of elements of $S$ and their inverses.
    We say that $S$ \textbf{generates} $G$ and write $G=(s)$. We call a \textbf{relation} of the
    group any equation satisfied by the generators of $G$.
\end{definition}

\begin{example}
    \begin{enumerate}[label=(\arabic*)]
        \item If $G$ is any group, and  $S$ is the set of generators of  $G$ with  $|S|=1$, then
            $G$ is a cyclic group.

        \item In $(\Z,+)$, $\{1\}$ generates $\Z$, so  $\Z=\{1\}$.

        \item In $D_{2n}$, we have that $\{\tau,\rho\}$ generate all of $D_{2n}$, so
            $D_{2n}=(\tau,\rho)$. and $D_{2n}$ has the relations $\tau^2=\iota$,  $\rho^n=\iota$ and
             $\rho\tau=\tau\rho^{-1}$.
    \end{enumerate}		
\end{example} 

We can also write $G$ in terms of just its generators and relations as  $G=\{S:R_1, \dots, R_m\}$,
where $R_i$ is a relation of the generators for  $1 \leq i \leq m$. Then for the dihedral group, we
have  $D_{2n}=\{\tau,\rho:\tau^2=\rho^n=\iota, \text{ } \rho\tau=\tau\rho^{-1}\}$


% Chapter Template

\chapter{Binomial Coefficients} % Main chapter title

\label{Chapter 2} % Change X to a consecutive number; for referencing this chapter elsewhere, use \ref{ChapterX}

%% to include section files use the \input{} command.

\begin{definition}
    Let $n \in \N$ and  $k \in \Z$. We define the \textbf {binomial coefficient} of $n$ \textbf
    {choose} $k$ to be 
        \begin{equation}
        \binom{n}{k}=\begin{cases}
            \frac{n!}{k!(n-k)!} & \text{, if } 0 \leq k \leq n \\
            0                   & \text{, if } k>n \text{ or } k<0 \\
                    \end{cases}
        \end{equation} 
    and where $\binom{n}{0}=\binom{n}{n}=1$.
\end{definition}

The solution to problem $3$ is a sufficient proof for the relation; and the fact that
$\binom{n}{k}=0$ for $n<k$ and  $k<0$ is evident since there can be no  $k$ element subsets of an
$n$ element set under those conditions. What follows are some fundamental lemmas about the binomial
coefficient.

\begin{lemma}[Symmetry]\label{2.0.1}
    $\binom{n}{k}=\binom{n}{n-k}$
\end{lemma}
\begin{proof}
    $\binom{n}{k}=\frac{n!}{k!(n-k)!}=\frac{n!}{(n-k)!(n-(n-k))!}=\binom{n}{n-k}$.
\end{proof}

\begin{lemma}[Pascal's Lemma]\label{2.0.2}
    If $n \in \N$ and  $k \in Z$ then 
        \begin{equation}
            \binom{n}{k}=\binom{n-1}{k}+\binom{n-1}{k-1}.
        \end{equation} 
\end{lemma}
\begin{proof}
    For $k<0$ or  $k>n$, the result is obvious by definition. Now suppose that $0 \leq k \leq n$,
    then
    $\binom{n-1}{k}+\binom{n-1}{k-1}=\frac{(n-1)!}{k!(n-k-1)!}+\frac{(n-1)!}{(k-1)!(n-k)!}=(n-1)!\frac{(n-k)+k}{k!(n-k)!}=\binom{n}{k}$.
\end{proof}
\begin{remark} 
    We can use Pascal's lemma to construct Pascal's triangle in figure \ref{fig:2.1}.
\end{remark}

\begin{figure} 
    \centering
        \begin{tabular}{>{$n=}l<{$\hspace{12pt}$}*{13}{c}}
            &&&&&&&1&&&&&&\\
            &&&&&&1&&1&&&&&\\
            &&&&&1&&2&&1&&&&\\
            &&&&1&&3&&3&&1&&&\\
            &&&1&&4&&6&&4&&1&&\\
            &&1&&5&&10&&10&&5&&1&\\
            &1&&6&&15&&20&&15&&6&&1
        \end{tabular}
        \caption{Pascal's triangle up to $n=6$.}
        \label{fig:2.1}
\end{figure}

\begin{theorem}[The Binomial Theorem]\label{2.0.3}
    If $n \in \N$, then
        \begin{equation}
            (x+y)^n=\sum_{k}{\binom{n}{k}x^ky^{n-k}}
        \end{equation} 
\end{theorem}
\begin{proof}
    We use induction on $n$. Notice that  $(x+y)^0=1=\binom{0}{0}x^0y^0$, and
    $(x+y)^1=x+y=\binom{1}{0}x^0y^1+\binom{1}{1}x^1y^0$
\end{proof}
\begin{proof}
    We use induction on $n$. Notice that  $(x+y)^0=\binom{0}{0}x^0y^0$ and
    $(x+y)^1=x+y=\binom{1}{0}x^0y^1+\binom{1}{1}x^1y^0$. Now suppose that the theorem holds for $n
    \geq 1$. Then
    $(x+y)^{n+1}=(x+y)\sum{\binom{n}{k}x^ny^{n-k}}=
    \sum{\binom{n}{k}x^{k+1}y^{n-k}}+\sum{\binom{n}{k}x^ky^{n+1-k}}=\sum{(\binom{n}{k-1}\binom{n}{k})x^ky^{n+1-k}}=
    \sum{\binom{n+1}{k}x^ky^{n+1-k}}$, by Pascal's lemma.
\end{proof}
\begin{corollary}
    $\sum_k{\binom{n}{k}}=2^n$ and there are exactly $2^n$ subsets of an  $n$ element set.		
\end{corollary}
\begin{proof}
    Exapand $(1+1)^n$, also notice that since there are $\binom{n}{k}$ possible $k$ element subsets
    of a given  $n$ element set, then  all we need to do is  take the sum of $\binom{n}{k}$ over
    $k$.
\end{proof}
\begin{corollary}
    $\sum_{k}{(-1)^k\binom{n}{k}}=\begin{cases}
                                    0 & , \text{ if } n \geq 1
                                    1 & , \text{ if } n=0
                                \end{cases}$
\end{corollary}
\begin{proof}
    Expand $(-1+1)^n$.
\end{proof}

\begin{lemma}\label{2.0.4}
    If $m,n \in \N$, then  $\sum_{k}{\binom{k}{m}}=\binom{n+1}{m+1}$.
\end{lemma}
\begin{proof}
    For $n=0$, the result follows,  $\binom{0}{0}=1=\binom{1}{1}$, and if $m>0$, then each side is
    $0$. Now suppose for $n \geq 0$ that the lemma holds. Then
    $\sum_{k=0}^{n+1}{\binom{k}{m}}=\binom{n+1}{m}+\binom{n+1}{m+1}=\binom{n+1}{m+1}$ by Pascal's
    lemma.
\end{proof}

\begin{theorem}[Vandermonde's Convolution]\label{2.0.5}
    If $m,n \in \N$ and  $l,p \in \Z$, then
        \begin{equation}
            \sum_{k}{\binom{m}{p+k}\binom{n}{l-k}}=\binom{m+n}{l+p}
        \end{equation} 
\end{theorem}
\begin{proof}
    By induction on $n$, for  $n=0$,  $\binom{m}{l+p}=\sum{\binom{m}{p+k}}$. Now suppose that the
    theorem holds for $n \geq 0$, then by Pascal's lemma we have that
    $\binom{m+n+1}{p+l}=\binom{m+n}{p+l}+\binom{m+n}{p+l-1}=\sum{\binom{m}{p+k}\binom{n}{l-k}}=
    \binom{m}{p+k}\binom{n+1}{0}+\sum{\binom{m}{p+k}\binom{n+1}{l-k}}=\sum{\binom{m}{p+k}\binom{n+1}{l-k}}$.
\end{proof}

% Chapter Template

\chapter{Multinomial Coefficients} % Main chapter title

\label{Chapter 3} % Change X to a consecutive number; for referencing this chapter elsewhere, use \ref{ChapterX}

%% to include section files use the \input{} command.



\end{document}
