%----------------------------------------------------------------------------------------
%	SECTION 1.1
%----------------------------------------------------------------------------------------

\section{The Riemann-Stieltjes Integral.}

\begin{definition}
    Let $[a,b]$ be an interval. A \textbf{partition} of $[a,b]$ is a set of
    points  $P=\{x_0,x_1, \dots, x_n\}$ such that $a=x_0<x_1< \dots <x_n=b$, and
    we write $\Delta{x_i}=x_i-x_{i-1}$. Now let  $f:[a,b] \rightarrow \R$ be a bounded 
    realvalued function, and for each partition $P$ of  $[a,b]$ let
    $M_i=\sup{f}$ and  $m_i=\inf_{f}$ for all  $x_{i-1} \leq x \leq x_i$. We
    define the \textbf{upper Riemann sum} and the \textbf{lower Riemmann sum} to
    of  $f$ with respect to be:
       \begin{align}
           U(f,P) &= \sum_{i=1}^{n}{M_i\Delta{x_i}} \\
           L(f,P) &= \sum_{i=1}^{n}{m_i\Delta{x_i}}
       \end{align}   
    respectively. We also define the \textbf{upper Riemann integral} and the
    \textbf{lower Riemann integral} of $f$ over  $[a,b]$ to be: 
       \begin{align}
           \bar{\int_{a}^{b}}{f(x)}dx=\inf{U(f,P)} \\
           \bbar{\int_{a}^{b}}{f(x)}dx=\sup{L(f,P)}
       \end{align}
    Respectively.
    
    If $\bar{\int_{a}^{b}}{f}=\bbar{\int_{a}^{b}}{f}$, then we say that  $f$ is
    \textbf{Riemann integrable} on  $[a,b]$, and we its value the
    \textbf{Riemann integral}, and denote it to be: 
        \begin{equation}
            \int_{a}^{b}{f(x)}dx=\bar{\int_{a}^{b}}{f(x)}dx=\bbar{\int_{a}^{b}}{f(x)}dx
        \end{equation} 
\end{definition}

\begin{lemma}\label{7.1.1}
    $\bar{\int_{a}^{b}}{f}$,and  $\bbar{\int_{a}^{b}}{f}$ are defined for every
    bounded realvalued function $f$ over  $[a,b]$.
\end{lemma}
\begin{proof}
    Let $f$ be bounded on  $[a,b]$, then there are  $m$ and  $M$ such that  $m
    \leq f \leq M$ for all  $a \leq x \leq b$. Now let  $P$ be a partition of
    $[a,b]$. Since  $\inf{f} \leq \sup{f}$, we have that $m \leq m_i=\inf{f}
    \leq M_i=\sup{f} \leq M$, thus  $m(b-a) \leq L(f,P) \leq U(f,P) \leq
    M(b-a)$, hence $L$ and  $U$ form a bounded set, and we are done.
\end{proof}
\begin{corollary}
    $L(f,P) \leq U(f,P )$ for every bounded function $f$.
\end{corollary}

Now the question of the integrability of $f$ is a very delicate matter, and
requires a closer scrutiny on the concepts of upper and lower sums. Infact, it
turns out that the Riemann integral is a consequence of a more general class of
integrals. Developng this more general situation will allow us to discern facts
about the Riemann integral.

\begin{definition}
    Let $\alpha$ be a bounded monontonically increasing function on  $[a,b]$,
    and let  $P$ be a partition of  $[a,b]$ and let
    $\Delta{\alpha_i}=\alpha(x_i)-\alpha(x_{i-1})$. For any realvalued, bounded
    function on  $[a,b]$, defined the \textbf{upper sum} and the \textbf{lower
    sum} of $f$ with respect to  $P$ and  $\alpha$ to be: 
        \begin{align}
            U(f,P,\alpha) &= \sum_{i=1}^{n}{M_i\Delta{\alpha_i}} \\		
            L(f,P,\alpha) &= \sum_{i=1}^{n}{m_i\Delta{\alpha_i}}
        \end{align}
        Where $M_i=\sup{f}$ and  $m_i=\inf{f}$ for all  $x_{i-1} \leq x \leq
        x_i$, and again, define the \textbf{upper integral} and  \textbf{lower
        integral} of $f$ with respect to  $\alpha$ on  $[a,b]$ to be:
            \begin{align}
                \bar{\int_{a}^{b}}{f(x)}d\alpha=\inf{U(f,P,\alpha)} \\
                \bbar{\int_{a}^{b}}{f(x)}d\alpha=\sup{L(f,P,\alpha)}	
            \end{align}
            If $\bar{\int_{a}^{b}}{f}d\alpha=\bbar{\int_{a}^{b}}{f}d\alpha$, we
            call the value:
                \begin{equation}
                    \int_{a}^{b}{f(x)}d\alpha=\bar{\int_{a}^{b}}{f(x)}d\alpha=\bbar{\int_{a}^{b}}{f(x)}d\alpha
                \end{equation}
            the \textbf{Riemann-Stieltjes integral} of $f$ with respect to  $f$
            on  $[a,b]$. If such an integra exists, we say that  $f$ is
            \textbf{integrable} with respect to  $\alpha$ on  $[a,b]$.
\end{definition}

\begin{example}
    Let $\alpha(x)=\alpha$, be defined over  $[a,b]$. Then  $\alpha$ is
    monontonically increasing, and our definititions reduces to those for the
    Riemann integral. Here  $U(f,P,x)=U(f,P)$ and  $L(f,P,x)=L(F,P)$.
\end{example} 

We are now in a position to investigate the properties of integrability, in the
Riemann-Stieltjes sense.

\begin{definition}
    Let $a,b]$ be an interval, and let  $P$ and  $Q$ be partitions of  $[a,b]$.
    We say that  $Q$ is a \textbf{refinment} of $P$ if  $P \susbeteq Q$, and we
    also say that $Q$ is \textbf{finer} than  $P$. Now if neither  $P$ nor  $Q$
    is a refinment of the other, we say that the two partitions are
    \textbf{noncomparable}.
\end{definition}

\begin{lemma}\label{7.1.2}
    Let $P$ and  $Q$ be partitions of and interval $[a,b]$, then  $P \cup Q$ is
    a partition of  $[a,b]$, and is a refinment of both  $P$ and  $Q$.
\end{lemma}
\begin{proof}
    If $P$ is a refinment of  $Q$, or viceversa, then we are done; so suppose
    that  $P$ and  $Q$ are noncomparable. Let  $P=\{x_0,x_1, \dots, x_n\}$ and
    $Q=\{y_0,y_1, \dots, y_m\}$ with $a=x_0<x_1< \dots x_n=b$ and $a=y_0<y_1<
    \dots y_m=b$. Then $P \cup Q=\{x_0,y_0,x_1,y_1, \dots, x_n,y_m\}$ and
    $a=x_0=y_0 < x_1,y_1 < \dots <x_n=y_m=b$, thus $P \cup Q$ is a partition of
    $[a,b]$, that it is a refinment of  $P$ and  $Q$ follows trivially.
\end{proof}

\begin{theorem}\label{7.1.3}
    Let $\alpha$ be monontonically increasing, and bounded on  $[a,b]$, and let
    $P$ and  $Q$ be partitions of  $[a,b]$. If  $Q$ is a refinment of  $P$, then
    $L(f,P,\alpha) \leq L(f,Q,\alpha)$ and  $U(f,Q,\alpha) \leq U(f,P,\alpha)$.
\end{theorem}
\begin{proof}
    Let $Q=P \cup \{x'\}$ and suppose that  $x_{i-1} \leq x' \leq x_i$.  Let 
    $w_1=\inf{f}$ for $x_{i-1} \leq x \leq x'$ and let $w_2=\inf{f}$ for $x'
    \leq x \leq x_i$. Then $m_i \leq w_1,w_2$, thus
    $L(f,Q,\alpha)-L(f,P,\alpha)=(w_1-m_i)(\alpha(b)-\alpha(a))-(w_2-m_i)(\alpha(b)-\alpha(a))
    \geq 0$, we are done. The proof is analogous for $U$.
\end{proof}
\begin{corollary}
    $L(f,P,\alpha)$ is monontonically increasing and  $U(f,P,\alpha)$ is
    monontonically decreasing.		
\end{corollary}
\begin{proof}
    We note that if $Q$ is a refinment of  $P$, then  $|P| \leq |Q|$, the result
    follows by direct application.
\end{proof}
\begin{remark}
    If $Q$ contains  $k$ more points than  $P$, we can repeat the proof
    inductively.
\end{remark}

\begin{theorem}\label{7.1.4}
    $\bbar{\int}{f}d\alpha \leq \bar{\int}{f}d\alpha$.
\end{theorem}
\begin{proof}
    Let $P=P_1 \cup P_2$ for partitions $P_1$ and $P_2$ of $[a,b]$. By theorem
    \ref{7.1.3} and lemma  \ref{7.1.1}, we have:
        \begin{equation}
            L(f,P_1,\alpha) \leq L(f,P_2,\alpha) \leq U(f,P_2, \alpha) \leq
            U(f,P_1, \alpha)
        \end{equation}
        Fixing $P_2$ and taking the supremum over all $P_1$, we get
        $\bbar{\int}{f} \leq U(f,P_2,\alpha)$, the infimum over $P_2$ we
        get $\bbar{\int}{f} \leq \bar{\int}{f}$
\end{proof}
