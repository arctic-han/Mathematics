%----------------------------------------------------------------------------------------
%	SECTION 1.1
%----------------------------------------------------------------------------------------

\section{Espacios M\'etricos.}

\begin{definition}
    Una \textbf{M\'etrica} sobre un conjunto $X$ es ina funcion  $d:X \times X \rightarrown \R$ tal
    que para toda $x,yz \in X$:
        \begin{enumerate}[label=(\arabic*)]
            \item $d(x,y) \geq 0$ y $d(x,y)=0$ si y solo si $x=y$.

            \item $d(x,y)=d(y,x)$.

            \item $d(x,z) \leq d(x,y)+d(y,z)$ (La Desigualdad Triangular).
        \end{enumerate}
        Si $d$ es una m\'etrica sobre el conjunto $X$, entonces decimos que el par ordenado  $(X,d)$
        es un \textbf{espacio m\'etrico}, y que $ d(x,y)$ es la \textbf{distancia} entre $x$ y y.
\end{definition}

\begin{example}
    Sea $X=\R$ y  $d=|\cdot|$, entonces $(\R,|\cdot|)$ es una espacio m\'etrico. Para $X=\R^2$ y
    $d=||\cdot||$,  $(\R^2,||\cdot||)$ tambien es un espacio metrico.		
\end{example} 

\begin{example}[La M\'etrica Discreta]
    Sea $X$ cualquier conjunto, y sea  $d(x,y)=\begin{cases} 1 & x \neq y \\ 0 & x=y \\\end{cases}$ 
    Vemos que las propiedades $(1)$ y $(2)$ estan satisfecho. Tambien vemos que para $x,y,z \in X$
    que  $d(x,z)=1,0$ y que $d(x,y),d(y,x)=1,0$. Pues $d(x,y)+d(y,z)=2,1,0$, pues en todo caso
    $d(x,z) \leq d(x,y)+d(y,z)$. $(X,d)$ es un espacio metrico.
\end{example} 
