\documentclass[12pt]{book}
 
\usepackage[margin=1in]{geometry}
\usepackage[utf8]{inputenc}
\usepackage{verbatim}
\usepackage{pgfplots}
    \pgfplotsset{compat=1.12,}
\usepackage{enumitem}
\usepackage{amsmath,amsfonts,amsthm,amssymb,graphicx,mathtools,hyperref}
\usepackage{wrapfig}
\usepackage{pdfpages}
\usepackage[export]{adjustbox}
\usepackage{tikz}
\renewcommand\qedsymbol{$\blacksquare$}
\usetikzlibrary{positioning}
\input{Library/Fonts/BlackboardBold}
\input{Library/Fonts/MathCalligraphy}
\newcommand{\ita}[1]{\textit{#1}}
\newcommand{\com}[2]{#1\backslash#2}
\newcommand{\oneton}{\{1,2,3,...,n\}}
\newcommand\idea[1]{\begin{gather*}#1\end{gather*}}
\newcommand\ef{\ita{f} }
\newcommand\eff{\ita{f}}
\newcommand\proofs[1]{\begin{proof}#1\end{proof}}
\newcommand\inv[1]{#1^{-1}}
\newcommand\setb[1]{\{#1\}}
\newcommand\en{\ita{n }}
\newcommand{\vbrack}[1]{\langle #1\rangle}
\DeclareMathOperator{\ord}{ord}

\theoremstyle{plain}
\newtheorem{theorem}{Theorem}[section]
\newtheorem{lemma}[theorem]{Lemma}
\newtheorem{proposition}[theorem]{Proposition}
\newtheorem*{corollary}{Corollary}

\theoremstyle{plain} % just in case the style had changed
\newcommand{\thistheoremname}{}
\newtheorem*{genericthm}{\thistheoremname}
\newenvironment{namedtheorem}[1]
  {\renewcommand{\thistheoremname}{#1}%
   \begin{genericthm}}
  {\end{genericthm}}

\theoremstyle{definition}
\newtheorem*{definition}{Definition}
\newtheorem{conjecture}{Conjecture}
\newtheorem{example}{Example}[chapter]
\newtheorem*{HW}{Homework}

\theoremstyle{remark}
\newtheorem*{remark}{Remark}
\newtheorem*{claim}{Claim}
\newtheorem*{note}{Note}

\renewcommand*{\proofname}{Proof}


 
\title{Examen 1}
\author{Alec Zabel-Mena\\ 801-16-9720 \\
alec.zabel@upr.edu}
\date{\today}

\begin{document}

\maketitle

\includepdf[pages=-]{tarea1_mate6540.pdf}

\begin{enumerate}[label=(\arabic*)]
    \item Let $d$ be a metric on  $X$, and define  $d':X \times X \rightarrow \R$ by
        $d'(x,y)=\frac{d(x,y)}{1+d(x,y)}$. We would like to show that $d'$ is a metric on  $X$.

        \begin{lemma}\label{0.0.1}
            $d'$ is a metric on  $X$.
        \end{lemma}
        \begin{proof}
            We have that $d(x,y) \geq 0$, and $d(x,y)=0$ if and only if $x=y$. Then notice that
            $1+d(x,y) \geq 1 > 0$, and that $ \frac{d(x,y)}{1+d(x,y)} \geq 0$. Moreover,
            $\frac{d(x,y)}{1+d(x,y)}=0$ if and only if $x=y$. Notice as well that
                \begin{equation*}
                    d'(x,y)=\frac{d(x,y)}{1+d(x,y)}=\frac{d(y,x)}{1+d(y,x)}=d'(y,x)
                \end{equation*}

            Finally, letting $x,y,z \in X$, we have:
                \begin{align*}
                    d'(x,z) &= \frac{d(x,z)}{1+d(x,z)} \\
                            &\leq \frac{d(x,y)+d(y,z)}{1+d(x,y)+d(y,z)} \\
                            &\leq \frac{d(x,y)}{1+d(x,y)}+\frac{d(y,z)}{1+d(y,z)} \\
                            &= d'(x,y)+d'(y,z)
                \end{align*}
            Hence, the triangle inequality is satisfied, and $d'$ is a metric on  $X$.
        \end{proof}

        We have a further property of this metric.

        \begin{corollary}
            $d'$ is a bounded metric.           		
        \end{corollary}
        \begin{proof}
            Consider the function $f:\R \rightarrow \R$ defined by  $f(x)=\frac{x}{1+x}$, which is
            continous on an interval $[a,b]$ and differentiable on $(a,b)$. Then by the mean value
            theorem, we have that there is a point $x \in (a,b)$for which
                \begin{equation*}
                    \frac{b}{1+b}-\frac{a}{1+a}=\frac{b-a}{x^2}
                \end{equation*}
            Then we have that
                \begin{equation*}
                    \frac{a+b}{1+a+b+ab}=\frac{b-a}{x^2} \leq \frac{a+b+2ab}{1+a+b+ab}
                \end{equation*}
                Hence we have that $f(a+b) \leq f(a)+f(b)$. Then composing $f$ with  $d$, we get
                $d'=f \circ d$, and so this shows that  $d'$ is a bounded metric.
        \end{proof}

        \begin{corollary}
            $d'$ induces the same topology as  $d$.		
        \end{corollary}
        \begin{proof}
            Consider the $\epsilon$-ball $B_d(x,\epsilon)$. By definition, we have that $d'(x,y)
            \leq d(x,y)$, Then $d'(x,y)=\frac{d(x,y)}{1+d(x,y)} \leq d(x,y)<\epsilon$, it follows
            that there is a $\delta$-ball  $B_{d'}(x,\delta) \subseteq B_d(x,\epsilon)$.

            On the other hand, we have $d'(x,y)+d(x,y)d'(x,y)=d(x,y)$, suppose then, that $y \in
            B_d(x,\epsilon)$, then $d'(x,y)+d(x,y)d'(x,y)<\epsilon$, then we also have that $d'(x,y)<\epsilon$; 
            that is to say $B_d(x,\epsilon) \subeteq B_{d'}(x,\epsilon)$.
        \end{proof}

    \item \begin{lemma}\label{0.0.2}
            Let $\{\Tc_\alpha\}$ be a collection of topological spaces on $X$. Then
            $\Tc=\bigcap{\Tc_\alpha}$ is also a topological space on $X$.
          \end{lemma}
        \begin{proof}
            Let $\{\Tc_\alpha\}$ be a collection of topological spaces on a set $X$, and consider
        $\Tc=\bigcap{\Tc_\alpha}$. Clearly, $\emptyset, X \in \Tc$, as $\emptyse, X \in \Tc_\alpha$
        for all $\alpha$. Now consider a collection $\{U_\alpha\}$, where $U_\alpha$ is open in  $X$
        under  $\Tc_\alpha$. Then  $u_\alpha \in \Tc_\alpha$, hence so is $\bigcup{U_\alpha}$. This
        implies that $\bigcup{U_\alpha} \in \Tc$.

        Similarly let $\{U_i\}_{i=1}^{n}$ be a collection with $U_i$ open in  $X$ under  $\Tc_i$,
        for  $1 \leq i \leq n$, then by similar reasoning,  $U_i \in \Tc_i$, and hence so is
        $\bigcap_{i=1}^{n}{U_i}$, hence $\bigcap_{i=1}^{n}{U_i} \in \Tc$. This makes $\Tc$ a
        topology on  $X$.
        \end{proof}
        \begin{remark} 
            Consider $X=\{a,b,c\}$ and the topologies $\Tc_1=\{\emptyset,\{a\},X\}$ and
            $\Tc_2=\{\emptyset, \{b\}, X\}$. Then $\Tc_1 \cup \Tc_2=\{\emptyset, \{a\}, \{b\}, X\}$,
            which is not a topology for it does not contain $\{a,b\}$. In general,
            $\Tc=\bigcup{\Tc_\alpha}$ is not a topology on arbitrary $X$.
        \end{remark}

    \item \begin{lemma}\label{0.0.3}
            Let $\{\Tc_\alpha\}$ be a collection of topologies on $X$. Then there exists a unique
            coarsest topology on $X$ containing all  $\Tc_\alpha$, and there exists a unique finest
            toplogy on  $X$ contained in all of  $\Tc_\alpha$.
      \end{lemma}
      \begin{proof}
          From lemma \ref{0.0.2}, we have that $\bigcap{\Tc_\alpha}$ is a topology on $X$. Now
          define  $\Tc_s$ to be the topology on  $X$ generated by the subbasis
          $\Sc=\bigcup{\Tc_\alpha}$ (which is a subbasis by definition), and suppose there is
          another topology $\Tc_s'$ on  $X$ generated by  $\Sc$, then $\Sc \in \Tc_s'$. Now take $U
          \in \Tc_s$ with  $U \not\subseteq \Sc$, then either  $U=\bigcup{U_\alpha}$ or
          $U=\bigcap_{i=1}^{n}{U_i}$ (with $U_\alpha$, $U_i$ open in X). Since $U \not\subseteq
          \Sc$, and if  $U=\bigcup{U_\alpha}$, then $U \notin \Tc_s'$, a contradiction. On the
          otherhand, if $U=\bigcap_{i=1}^{n}{U_i}$, then  $U \subseteq \Sc$, another contradiction.
          Hence $\Tc_s$ is the coarsest such topology.

          Now let $\Tc_l=\bigcap{\Tc_\alpha}$, and suppose there is another topology $\Tc_l'$ for
          which $\Tc_l \subseteq \Tc_l'$ and $\Tc_l' \subseteq \Tc_\alpha$ for all  $\alpha$. Then
          take  $U_\alpha \in \Tc'$, then  $U_\alpha \in \Tc_\alpha$, for all $\alpha$ 
          hence  $U_\alpha \in \bigcap{\Tc_\alpha}=\Tc_l$, thush $\Tc_l' \subseteq \Tc_l$. 
          Then $\Tc_l$ is the finest such topology.
      \end{proof}

  \item \begin{lemma}\label{0.0.4}
      The collection $\Cc=\{[a,b): a,b \in \Q \text{ and } a<b\}\}$ forms a basis generating a
      topology different from the lower limit topology $\R_l$.
    \end{lemma}
    \begin{proof}
        First, consider an element $[c,d) \in \Cc$, by the nested interval theorem, we have that
        there is a $[c',d') \subseteq [c,d)$, then taking $\Cc'=\{[c',d') \in \Cc: [c',d') \subseteq [c,d)\}$
        We get that for $U \neq \emptyset$ open in  $\R_l$ that $U=\bigcap_{C \in \Cc'}{C}$. Thus $\Cc$
        forms a basis for a topology on  $\R_l$.

        Now consider $\R_l$ in the lower limit topology, and let  $[a,b) \in \R_l$ be a basis element.
        Since $\Q$ is dense in  $\R$, there exists a  $[c,d) \subseteq [a,b)$; however, $\R$ is not
        dense in  $\Q$, so $[a,b) \subseteq [c,d)$ does not hold, for $c,d \in \Q$. So  $\Cc$ forms
        a basis that generates a topology different from that of  $\R_l$.
    \end{proof}

\item 
    \begin{lemma}\label{0.0.5}
        Let $X=\{f:f:[0,1] \rightarrow [0,1]\}$, and for each subset $A \subseteq [0,1]$, let
    $B_A=\{f \in X: f(x)=0 \text{for all } x \in A\}$, and define $\Bc=\{B_A:A \subseteq [0,1]\}$.
    Then $\Bc$ is a basis for a topology on  $X$.
    \end{lemma}
    \begin{proof}
        Since $A \subeteq [0,1]$, by the nested interval theorem, there is an interval $C$ such that
        $C \subseteq A \subseteq [0,1]$. Then for some $x \in C$, we have  $f(x)=0$, that is $B_C
        \subseteq B_A$. Now take  $\Bc'=\{B_C:C \subseteq A\}$, then $\Bc' \subseteq \Bc$, and given
         $U \neq \emptyset$ open in  $X$, since  $x$ was arbitrary,  $U=\bigcup_{B \in \Bc'}{B}$.
         Therefore $\Bc$ is a basis for a topology on  $X$.
    \end{proof}
\end{enumerate

\end{document}
